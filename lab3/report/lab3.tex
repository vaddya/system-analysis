\documentclass[a4paper,14pt]{extarticle}

\usepackage[utf8x]{inputenc}
\usepackage[T1]{fontenc}
\usepackage[russian]{babel}
\usepackage{hyperref}
\usepackage{indentfirst}
\usepackage{here}
\usepackage{array}
\usepackage{graphicx}
\usepackage{caption}
\usepackage{subcaption}
\usepackage{chngcntr}
\usepackage{amsmath}
\usepackage{amssymb}
\usepackage[left=2cm,right=2cm,top=2cm,bottom=2cm,bindingoffset=0cm]{geometry}
\usepackage{multicol}
\usepackage{multirow}
\usepackage{titlesec}
\usepackage{listings}
\usepackage{listingsutf8}
\usepackage{color}
\usepackage{enumitem}
\usepackage{cmap}
\usepackage{url}

\definecolor{green}{rgb}{0,0.6,0}
\definecolor{gray}{rgb}{0.5,0.5,0.5}
\definecolor{purple}{rgb}{0.58,0,0.82}

\lstdefinelanguage{none}{}

\lstset{
	language={Python},
	inputpath={../},
	backgroundcolor=\color{white},
	commentstyle=\color{green},
	keywordstyle=\color{blue},
	numberstyle=\color{gray}\scriptsize\ttfamily,
	stringstyle=\color{purple},
	basicstyle=\lst@ifdisplaystyle\footnotesize\fi\ttfamily,
	breakatwhitespace=false,
	breaklines=true,
	captionpos=b,
	keepspaces=true,
	numbers=left,
	numbersep=5pt,
	showspaces=false,
	showstringspaces=false,
	showtabs=false,
	tabsize=4,
	frame=single,
	morekeywords={},
	deletekeywords={},
	extendedchars=true,
	columns=fullflexible,
	inputencoding=utf8/cp1251,
	literate=%
		{~}{{\raise.25ex\hbox{$\mathtt{\sim}$}}}{1}
}

\titleformat*{\section}{\large\bfseries} 
\titleformat*{\subsection}{\normalsize\bfseries} 
\titleformat*{\subsubsection}{\normalsize\bfseries} 
\titleformat*{\paragraph}{\normalsize\bfseries} 
\titleformat*{\subparagraph}{\normalsize\bfseries} 

\counterwithin{figure}{section}
\counterwithin{equation}{section}
\counterwithin{table}{section}
\newcommand{\sign}[1][5cm]{\makebox[#1]{\hrulefill}}
\newcommand{\equipollence}{\quad\Leftrightarrow\quad}
\newcommand{\no}[1]{\overline{#1}}
\newcommand{\code}[1]{\lstinline[language=none]|#1|}
\newcommand{\data}[2]{
\paragraph{Описание.} #2.
\begin{figure}[H]
	\centering
	\includegraphics[width=\linewidth]{#1}
	\caption{Результат работы на данных #1.}
\end{figure}
}

\graphicspath{{../pics/}}
\captionsetup{justification=centering,margin=1cm}
\def\arraystretch{1.3}
\setlength\parindent{5ex}
\titlelabel{\thetitle.\quad}

\setitemize{topsep=0em, itemsep=0em}
\setenumerate{topsep=0em, itemsep=0em}


\begin{document}

\begin{titlepage}
\begin{center}
	\textbf{Санкт-Петербургский Политехнический Университет \\Петра Великого}\\[0.3cm]
	\small Институт компьютерных наук и технологий \\[0.3cm]
	\small Кафедра компьютерных систем и программных технологий\\[4cm]
	
	\textbf{ОТЧЕТ}\\ \textbf{по расчетному заданию}\\[0.5cm]
	\textbf{<<Линейное программирование>>}\\[0.1cm]
	\textbf{Системный анализ и принятие решений}\\[8.0cm]
\end{center}

\begin{flushright}
	\begin{minipage}{0.4\textwidth}
		\begin{flushleft}
			\small \textbf{Работу выполнил студент}\\[3mm]
			\small группа 33501/4 \hspace*{6mm} Дьячков В.В.\\[5mm]
			
			\small \textbf{Преподаватель}\\[5mm]
		 	\small \sign[3cm] \hspace*{5mm} Сабонис С.С.\\[0.5cm]
		\end{flushleft}
	\end{minipage}
\end{flushright}

\vfill

\begin{center}
	\small Санкт-Петербург\\
	\small \today
\end{center}
\end{titlepage}

\addtocounter{page}{1}

\tableofcontents
%\newpage
\listoffigures
\listoftables
\newpage

\section{Техническое задание}

\begin{enumerate}
	\setlength{\itemsep}{0em}
	\item Записать необходимые условия оптимальности для задачи и решить задачу аналитически;
	\item Решить задачу методом релаксации;
	\item Решить задачу методом наискорейшего подъёма;
	\item Решить задачу методом Ньютона;
	\item Решить задачу методом сопряжённых градиентов;
	\item Решить задачу методом Бройдена.
\end{enumerate}

\section{Исходные данные}

\paragraph{Вариант 32}

Дана задача нелинейного программирования:
\begin{equation}
	\label{eq:main}
	\max f(X) = \max \left( -31 x_1^2 - 34 x_2^2 + 4 x_1 x_2 + 286 x_1 + 388 x_2 \right)
\end{equation}

\section{Аналитическое решение}

\paragraph{Необходимое условие максимума первого порядка} Пусть $f(X)$ дифференцируема в точке $X^* \in R^n$. Тогда если $X^*$ -- локальный экстремум, то $f'(X^*) = 0$. 

\paragraph{Достаточное условие максимума второго порядка} Пусть $f(X)$ дважды дифференцируема в точке $X^* \in R^n$. Тогда если $f'(X^*) = 0$, матрица $H(X*)$ отрицательно определена (полуопределена), то $X^*$ -- строгий (нестрогий) локальный экстремум.\\

Найдем матрицу Гессе $H(X)$:

\begin{multicols}{2}
	\centering
	$\dfrac{\partial f}{\partial x_1} = -62 x_1 + 4 x_2 + 286$\\
	$\dfrac{\partial f}{\partial x_2} = -68 x_2 + 4 x_1 + 388$
\end{multicols}

\begin{multicols}{3}
	\centering
	$\dfrac{\partial^2 f}{\partial x_1^2} = -62$\\
	$\dfrac{\partial^2 f}{\partial x_1 \partial x_2} = 4$\\
	$\dfrac{\partial^2 f}{\partial x_2^2} = -68$
\end{multicols}

\begin{equation*}
H(X) = H = 
\begin{pmatrix}
	-62 & 4 \\
	4 & -68
\end{pmatrix}
\end{equation*}


\paragraph{Критерий отрицательной определённости квадратичной формы} Для отрицательной определённости квадратичной формы необходимо и достаточно, чтобы угловые миноры чётного порядка её матрицы были положительны, а нечётного порядка — отрицательны.\\

Найдем главные миноры матрицы $H$:

\begin{multicols}{2}
\centering
$\Delta_1 = \begin{vmatrix} -62 \end{vmatrix} = -62$

$\Delta_2 = \begin{vmatrix}
	-62 & 4 \\
	4 & -68
\end{vmatrix} = 4200$
\end{multicols}
По критерию отрицательной определённости квадратичной формы, матрица $H$ отрицательно определена (выполнено достаточное условие максимума второго порядка). 

Найдем точку $X^*$:
\begin{equation*}
\begin{cases}
	-68 x_2 + 4 x_1 + 388 = 0 \\
	-62 x_1 + 4 x_2 + 286 = 0 
\end{cases}\Rightarrow
\begin{cases}
	x_1 = 17 x_2 - 97 \\
	-1050 x_2 + 6300 = 0
\end{cases}\Rightarrow
\begin{cases}
	x_1 = 5 \\
	x_2 = 6
\end{cases}
\end{equation*}

Следовательно максимум функции $f(X)$ достигается в точке $X^* = (5, 6)$: $f(X^*) = 1879$.

\section{Решение методом релаксации}

\section{Решение методом наискорейшего подъёма}

\section{Решение методом Ньютона}

\section{Решение методом сопряжённых градиентов}

\section{Решение методом Бройдена}

\end{document}