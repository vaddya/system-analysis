\documentclass[a4paper,14pt]{extarticle}

\usepackage[utf8x]{inputenc}
\usepackage[T1,T2A]{fontenc}
\usepackage[russian]{babel}
\usepackage{hyperref}
\usepackage{indentfirst}
\usepackage{here}
\usepackage{array}
\usepackage{graphicx}
\usepackage{caption}
\usepackage{subcaption}
\usepackage{chngcntr}
\usepackage{amsmath}
\usepackage{amssymb}
\usepackage{amsthm}
\usepackage{pgfplots}
\usepackage{pgfplotstable}
\usepackage[left=2cm,right=2cm,top=2cm,bottom=2cm,bindingoffset=0cm]{geometry}
\usepackage{multicol}
\usepackage{askmaps}
\usepackage{titlesec}
\usepackage{listings}
\usepackage{color}
\usepackage{enumerate}
\usepackage{hhline}
\usepackage{enumitem}
\usepackage{courier}
\usepackage{wrapfig}
\usetikzlibrary{arrows,automata}

\setitemize{itemsep=0em}
\setenumerate{itemsep=0em}

\theoremstyle{definition}

\pgfkeys{/pgf/number format/.cd,1000 sep={\,}}

\definecolor{green}{rgb}{0,0.6,0}
\definecolor{gray}{rgb}{0.5,0.5,0.5}
\definecolor{purple}{rgb}{0.58,0,0.82}

\lstset{
	language=python,
	backgroundcolor=\color{white},   
	commentstyle=\color{green},
	keywordstyle=\color{blue},
	numberstyle=\tiny\color{gray},
	stringstyle=\color{purple},
	basicstyle=\footnotesize\ttfamily,
	breakatwhitespace=false,
	breaklines=true,
	captionpos=b,
	keepspaces=true,
	numbers=left,
	numbersep=5pt,
	showspaces=false,
	showstringspaces=false,
	showtabs=false,
	tabsize=2,
	frame=single,
	inputpath={../code/}
}

\renewcommand{\le}{\ensuremath{\leqslant}}
\renewcommand{\leq}{\ensuremath{\leqslant}}
\renewcommand{\ge}{\ensuremath{\geqslant}}
\renewcommand{\geq}{\ensuremath{\geqslant}}
\renewcommand{\epsilon}{\ensuremath{\varepsilon}}
\renewcommand{\phi}{\ensuremath{\varphi}}
\renewcommand{\thefigure}{\arabic{figure}} 	
\newcommand{\norm}[1]{\left\lVert#1\right\rVert}
\newcommand*\sfrac[2]{{}^{#1}\!/_{#2}}

%\titleformat*{\section}{\large\bfseries} 
\titleformat*{\subsection}{\normalsize\bfseries} 
\titleformat*{\subsubsection}{\normalsize\bfseries} 
\titleformat*{\paragraph}{\normalsize\bfseries} 
\titleformat*{\subparagraph}{\normalsize\bfseries} 

\counterwithin{figure}{section}
\counterwithin{equation}{section}
\counterwithin{table}{section}
\newcommand{\sign}[1][5cm]{\makebox[#1]{\hrulefill}}
\graphicspath{{../pics/}}
\captionsetup{justification=centering,margin=1cm}
\setlength\parindent{5ex}
\def\arraystretch{1.3}
\def\tabcolsep{12pt}
%\titlelabel{\thetitle.\quad}

\DeclareMathOperator*{\argmin}{argmin}
\DeclareMathOperator*{\argmax}{argmax}

\begin{document}

\begin{titlepage}
\begin{center}
	\textbf{Санкт-Петербургский Политехнический Университет \\Петра Великого}\\[0.3cm]
	\small Институт компьютерных наук и технологий \\[0.3cm]
	\small Кафедра компьютерных систем и программных технологий\\[4cm]
	
	\textbf{ОТЧЕТ}\\ \textbf{по расчетному заданию}\\[0.5cm]
	\textbf{<<Построение моделей>>}\\[0.1cm]
	\textbf{Системный анализ и принятие решений}\\[8.0cm]
\end{center}

\begin{flushright}
	\begin{minipage}{0.4\textwidth}
		\begin{flushleft}
			\small \textbf{Работу выполнил студент}\\[3mm]
			\small группа 33501/4 \hspace*{6mm} Дьячков В.В.\\[5mm]
			
			\small \textbf{Преподаватель}\\[5mm]
		 	\small \sign[3cm] \hspace*{5mm} Сабонис С.С.\\[0.5cm]
		\end{flushleft}
	\end{minipage}
\end{flushright}

\vfill

\begin{center}
	\small Санкт-Петербург\\
	\small \the\year
\end{center}
\end{titlepage}

\addtocounter{page}{1}

\tableofcontents
\listoftables
\listoffigures
\newpage

\section{Техническое задание}

Задана сеть массового обслуживания, включающая $M = 4$ узла и источник с интенсивностью $\lambda_0$, заданы матрица передач и описание узлов как систем массового обслуживания (число каналов, интенсивность обслуживания). Необходимо:

\begin{enumerate}
	\item Построить граф сети;
	\item Выяснить, может ли данная сеть работать в установившемся режиме;
	\item Для заданной интенсивности $\lambda_0$:
		\begin{enumerate}[label*=\arabic*]
			\item Определить наиболее и наименее нагруженные узлы;
			\item Определить среднее число требований, среднее число ожидающих требований,
			среднее время пребывания и среднее время ожидания для каждого узла;
			\item Определить среднее число требований, среднее число ожидающих требований,
			среднее время пребывания и среднее время ожидания для всей сети;
			\item Результаты оформить в итоговой таблице.
		\end{enumerate}
\end{enumerate}

\section{Исходные данные}

\paragraph{Вариант 32}

\begin{itemize}
	\item $\lambda_0 = 3.5$
	\item Матрица передач:
	\begin{displaymath}
		\Pi = \begin{pmatrix}
			0 & 1 & 0 & 0 & 0 \\
			\sfrac{3}{4} & 0 & 0 & \sfrac{1}{4} & 0 \\
			0 & 0 & 0 & 0 & 1 \\
			0 & \sfrac{9}{17} & \sfrac{8}{17} & 0 & 0 \\
			0 & \sfrac{9}{14} & 0 & \sfrac{5}{14} & 0
		\end{pmatrix}
	\end{displaymath}
	\item 1 узел: система $M/M/2$, $\mu = 10$
	\item 2 узел: система $M/M/1$, $\mu = 3$
	\item 3 узел: система $M/M/2$, $\mu = 7$
	\item 4 узел: система $M/M/2$, $\mu = 7$
\end{itemize}

\section{Разомкнутая сеть массового обслуживания}

\subsection{Граф сети}

На рис. \ref{fig:net_graph} изображен граф сети массового обслуживания, составленный по матрице передач $\Pi$.
\vspace{-0.5cm}
\begin{figure}[H]
	\begin{center}
		\begin{tikzpicture}[->,>=stealth',shorten >=1pt,auto,node distance=3.5cm,
		semithick]
			\tikzstyle{every state}=[fill=white,draw=black,text=black]
			
			\node[state] (0) {$0$};
			\node[state] (1) [right of=0] {$1$};
			\node[state] (3) [right of=1] {$3$};
			\node[state] (2) [above of=3] {$2$};
			\node[state] (4) [above of=1] {$4$};
			
			\path 
			(0) edge [bend left] node {$ $} (1)
			(1) edge [bend left] node {$\sfrac{3}{4}$} (0)
			(1) edge [bend left] node {$\sfrac{1}{4}$} (3)
			(3) edge [bend left] node {$\sfrac{9}{17}$} (1)
			(3) edge [right] node {$\sfrac{8}{17}$} (2)
			(2) edge node {$ $} (4)
			(4) edge [left] node {$\sfrac{9}{14}$} (1)
			(4) edge node {$\sfrac{5}{14}$} (3);
		\end{tikzpicture}
		\caption{Граф сети}
		\label{fig:net_graph}
	\end{center}
\end{figure}
\vspace{-1cm}

\subsection{Интенсивности потоков в узлах}

Найдем коэффициенты передачи от источника к $i$-ому узлу $\alpha_i = \sfrac{\lambda_i}{\lambda_0}$:
\begin{displaymath}
	\begin{cases}
		\lambda_0 = \sfrac{3}{4} \lambda_1 \\
		\lambda_1 = \lambda_0 + \sfrac{9}{17} \lambda_3 + \sfrac{9}{14} \lambda_4 \\
		\lambda_2 = \sfrac{8}{17} \lambda_3 \\
		\lambda_3 = \sfrac{1}{4} \lambda_1 + \sfrac{5}{14} \lambda_4 \\
		\lambda_4 = \lambda_2
	\end{cases} \Leftrightarrow \begin{cases}
		\lambda_1 = \sfrac{4}{3} \lambda_0 \\
		\lambda_2 = \sfrac{56}{297} \lambda_0 \\
		\lambda_3 = \sfrac{119}{297} \lambda_0 \\
		\lambda_4 = \sfrac{56}{297} \lambda_0
	\end{cases} \Rightarrow \begin{cases}
		\alpha_1 = \sfrac{4}{3} \\
		\alpha_2 = \sfrac{56}{297} \\
		\alpha_3 = \sfrac{119}{297} \\
		\alpha_4 = \sfrac{56}{297}
	\end{cases}
\end{displaymath}

\subsection{Наличие установившегося режима}

Проверим выполнение условия наличия установившегося режима сети:
\begin{displaymath}
	\lambda_0 < \min \limits_{(j)} \left( \frac{m_j \cdot \mu_j}{\alpha_j} \right) = \min \limits_{(j)} \left( \beta_j \right)
\end{displaymath}

\begin{displaymath}
	\begin{cases}
		\beta_1 = \sfrac{2 \cdot 10 \cdot 4}{3} \approx 26.6 \\
		\beta_2 = \sfrac{1 \cdot 3 \cdot 297}{56} \approx 15.9 \\
		\beta_3 = \sfrac{2 \cdot 7 \cdot 297}{119} \approx 34.9 \\
		\beta_4 = \sfrac{2 \cdot 7 \cdot 297}{56} \approx 74.3
	\end{cases}	
\end{displaymath}

Условие установившегося режима выполняется, т.к. все $\beta_i < \lambda_0 = 3.5$.

\subsection{Разрезание сети}

Выполним разрезание сети и рассчитаем локальные показатели узлов сети, после чего произведем <<сборку>> сети, рассчитав интегральные показатели сети по теореме Джонсона. В таблице \ref{tab:results} приведены результаты рассчитанных показателей для отдельных узлов и для всей сети.
\begin{table}[H]
	\begin{center}
		\caption{Результаты}
		\label{tab:results}
		\def\tabcolsep{8pt}
		\begin{tabular}{|C{6cm}|c|c|c|c|c|}
			\hline
			& 1 узел & 2 узел & 3 узел & 4 узел & Вся сеть \\
			\hline
			Ср. число требований $\overline{n}$ & $0.4935$ & $0.0620$ & $0.2024$ & $0.0945$ & $0.8524$ \\
			\hline
			Ср. число ожидающих требований $\overline{l}$ & $0.0269$ & $0.0620$ & $0.0020$ & $0.0002$ & $0.0911$ \\
			\hline
			Ср. время пребывания $\overline{t_\text{с}}$ & $0.1058$ & $0.4273$ & $0.1443$ &  $0.1432$ & $0.3064$ \\
			\hline
			Ср. время ожидания $\overline{t_\text{ож}}$ & $0.0058$ & $0.2820$ & $0.0014$ & $0.0003$ & $0.0615$ \\
			\hline	
		\end{tabular}
	\end{center}
\end{table}

Из рассчитанных показателей можно сделать вывод, что по среднему число требований $\overline{n}$:
\begin{itemize}
	\item Наиболее нагруженным узлом является 1 узел;
	\item Наименее нагруженным узлом является 2 узел.
\end{itemize}

\end{document}