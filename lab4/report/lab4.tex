\documentclass[a4paper,14pt]{extarticle}

\usepackage[utf8x]{inputenc}
\usepackage[T1,T2A]{fontenc}
\usepackage[russian]{babel}
\usepackage{hyperref}
\usepackage{indentfirst}
\usepackage{here}
\usepackage{array}
\usepackage{graphicx}
\usepackage{caption}
\usepackage{subcaption}
\usepackage{chngcntr}
\usepackage{amsmath}
\usepackage{amssymb}
\usepackage{amsthm}
\usepackage{pgfplots}
\usepackage{pgfplotstable}
\usepackage[left=2cm,right=2cm,top=2cm,bottom=2cm,bindingoffset=0cm]{geometry}
\usepackage{multicol}
\usepackage{askmaps}
\usepackage{titlesec}
\usepackage{listings}
\usepackage{color}
\usepackage{enumerate}
\usepackage{hhline}
\usepackage{enumitem}
\usepackage{courier}
\usepackage{wrapfig}
\usetikzlibrary{arrows,automata}

\setitemize{itemsep=0em}
\setenumerate{itemsep=0em}

\theoremstyle{definition}

\pgfkeys{/pgf/number format/.cd,1000 sep={\,}}

\definecolor{green}{rgb}{0,0.6,0}
\definecolor{gray}{rgb}{0.5,0.5,0.5}
\definecolor{purple}{rgb}{0.58,0,0.82}

\lstset{
	language=python,
	backgroundcolor=\color{white},   
	commentstyle=\color{green},
	keywordstyle=\color{blue},
	numberstyle=\tiny\color{gray},
	stringstyle=\color{purple},
	basicstyle=\footnotesize\ttfamily,
	breakatwhitespace=false,
	breaklines=true,
	captionpos=b,
	keepspaces=true,
	numbers=left,
	numbersep=5pt,
	showspaces=false,
	showstringspaces=false,
	showtabs=false,
	tabsize=2,
	frame=single,
	inputpath={../code/}
}

\renewcommand{\le}{\ensuremath{\leqslant}}
\renewcommand{\leq}{\ensuremath{\leqslant}}
\renewcommand{\ge}{\ensuremath{\geqslant}}
\renewcommand{\geq}{\ensuremath{\geqslant}}
\renewcommand{\epsilon}{\ensuremath{\varepsilon}}
\renewcommand{\phi}{\ensuremath{\varphi}}
\renewcommand{\thefigure}{\arabic{figure}} 	
\newcommand{\norm}[1]{\left\lVert#1\right\rVert}
\newcommand*\sfrac[2]{{}^{#1}\!/_{#2}}

%\titleformat*{\section}{\large\bfseries} 
\titleformat*{\subsection}{\normalsize\bfseries} 
\titleformat*{\subsubsection}{\normalsize\bfseries} 
\titleformat*{\paragraph}{\normalsize\bfseries} 
\titleformat*{\subparagraph}{\normalsize\bfseries} 

\counterwithin{figure}{section}
\counterwithin{equation}{section}
\counterwithin{table}{section}
\newcommand{\sign}[1][5cm]{\makebox[#1]{\hrulefill}}
\graphicspath{{../pics/}}
\captionsetup{justification=centering,margin=1cm}
\setlength\parindent{5ex}
\def\arraystretch{1.3}
\def\tabcolsep{12pt}
%\titlelabel{\thetitle.\quad}

\DeclareMathOperator*{\argmin}{argmin}
\DeclareMathOperator*{\argmax}{argmax}

\begin{document}

\begin{titlepage}
\begin{center}
	\textbf{Санкт-Петербургский Политехнический Университет \\Петра Великого}\\[0.3cm]
	\small Институт компьютерных наук и технологий \\[0.3cm]
	\small Кафедра компьютерных систем и программных технологий\\[4cm]
	
	\textbf{ОТЧЕТ}\\ \textbf{по расчетному заданию}\\[0.5cm]
	\textbf{<<Построение моделей>>}\\[0.1cm]
	\textbf{Системный анализ и принятие решений}\\[8.0cm]
\end{center}

\begin{flushright}
	\begin{minipage}{0.4\textwidth}
		\begin{flushleft}
			\small \textbf{Работу выполнил студент}\\[3mm]
			\small группа 33501/4 \hspace*{6mm} Дьячков В.В.\\[5mm]
			
			\small \textbf{Преподаватель}\\[5mm]
		 	\small \sign[3cm] \hspace*{5mm} Сабонис С.С.\\[0.5cm]
		\end{flushleft}
	\end{minipage}
\end{flushright}

\vfill

\begin{center}
	\small Санкт-Петербург\\
	\small \the\year
\end{center}
\end{titlepage}

\addtocounter{page}{1}

\tableofcontents
%\newpage
\listoffigures
%\listoftables
\newpage

\section{Техническое задание}

\begin{enumerate}
	\setlength{\itemsep}{0em}
	\item Решить задачу методом Лагранжа при заданном ограничении;
	\item Решить задачу методом Била при заданных ограничениях;
	\item Решить задачу методом проекции градиента при заданных ограничениях;
	\item Решить задачу методом штрафных функций или методом барьерных функций при
заданном ограничении;
	\item Решить задачу методом возможных направлений при заданном ограничении.
\end{enumerate}

\section{Исходные данные}

\paragraph{Вариант 32}

Дана задача нелинейного программирования:
\begin{equation*}
	\max f(X) = \max \left( -31 x_1^2 - 34 x_2^2 + 4 x_1 x_2 + 286 x_1 + 388 x_2 \right)
\end{equation*}
Заданы коэффициенты $a_{ij}$:
\begin{center}
\begin{multicols}{5}
	$a_{11} = 7$\\
	$a_{21} = 10$\\
	$a_{31} = -1$\\
	$a_{41} = 0$\\
	$a_{51} = 0$\\
\end{multicols}
\begin{multicols}{5}
	$a_{12} = 12$\\
	$a_{22} = 8$\\
	$a_{32} = 0$\\
	$a_{42} = -1$\\
	$a_{52} = 1$\\
\end{multicols}
\end{center}
Заданы коэффициенты $b_i$:
\begin{multicols}{6}
	\centering
	$b_1 = 84$\\
	$b_2 = 80$\\
	$b_3 = 0$\\
	$b_4 = 0$\\
	$b_5 = 5$\\
	$b_6 = 400$\\
\end{multicols}
\noindent Заданы коэффициенты $d_i$:
\begin{multicols}{2}
	\centering
	$d_1 = 16$\\
	$d_2 = 25$\\
\end{multicols}

\section{Решение методом Лагранжа}

Решим задачу при ограничении:
\begin{equation*}
a_{51} x_1 + a_{52} x_2 = b_5 
\Longleftrightarrow
x_2 = 5
\end{equation*}

\section{Необходимые условия оптимальности при линейных ограничениях}

Запишем необходимые условия оптимальности для задачи при ограничениях:
\begin{equation*}
\begin{cases}
a_{11} x_1 + a_{12} x_2 \leq b_1 \\
a_{21} x_1 + a_{22} x_2 \leq b_2 \\
a_{31} x_1 + a_{32} x_2 \leq b_3 \\
a_{41} x_1 + a_{42} x_2 \leq b_4 \\
\end{cases}
\Longleftrightarrow
\begin{pmatrix}
	7 & 12 \\
	10 & 8 \\
	-1 & 0 \\
	0 & -1
\end{pmatrix}
\begin{pmatrix}
	x_1 \\
	x_2
\end{pmatrix} =
\begin{pmatrix}
	84 \\
	80 \\
	0 \\
	0
\end{pmatrix}
\end{equation*}

\section{Решение методом Била}

Решим задачу при ограничениях:
\begin{equation*}
\begin{cases}
a_{11} x_1 + a_{12} x_2 \leq b_1 \\
a_{21} x_1 + a_{22} x_2 \leq b_2 \\
a_{31} x_1 + a_{32} x_2 \leq b_3 \\
a_{41} x_1 + a_{42} x_2 \leq b_4 \\
\end{cases}
\end{equation*}

\section{Решение методом проекции градиента}

Решим задачу при ограничениях:
\begin{equation*}
\begin{cases}
a_{11} x_1 + a_{12} x_2 \leq b_1 \\
a_{21} x_1 + a_{22} x_2 \leq b_2 \\
a_{31} x_1 + a_{32} x_2 \leq b_3 \\
a_{41} x_1 + a_{42} x_2 \leq b_4 \\
\end{cases}
\end{equation*}

\section{Необходимые условия оптимальности при квадратичных ограничениях}

Решим задачу при ограничении:
\begin{equation*}
d_1 x_1^2 + d_2 x_2^2 \leq b_6
\Longleftrightarrow
16 x_1^2 + 25 x_2^2 \leq 400
\end{equation*}

\section{Решение методом барьерных/штрафных функций}

Решим задачу при ограничении:
\begin{equation*}
d_1 x_1^2 + d_2 x_2^2 \leq b_6
\end{equation*}

\section{Решение методом возможных направлений}

Решим задачу при ограничении:
\begin{equation*}
d_1 x_1^2 + d_2 x_2^2 \leq b_6
\end{equation*}

\end{document}