\documentclass[a4paper,14pt]{extarticle}

\usepackage[utf8x]{inputenc}
\usepackage[T1]{fontenc}
\usepackage[russian]{babel}
\usepackage{hyperref}
\usepackage{indentfirst}
\usepackage{here}
\usepackage{array}
\usepackage{graphicx}
\usepackage{caption}
\usepackage{subcaption}
\usepackage{chngcntr}
\usepackage{amsmath}
\usepackage{amssymb}
\usepackage[left=2cm,right=2cm,top=2cm,bottom=2cm,bindingoffset=0cm]{geometry}
\usepackage{multicol}
\usepackage{multirow}
\usepackage{titlesec}
\usepackage{listings}
\usepackage{listingsutf8}
\usepackage{color}
\usepackage{enumitem}
\usepackage{cmap}
\usepackage{url}

\definecolor{green}{rgb}{0,0.6,0}
\definecolor{gray}{rgb}{0.5,0.5,0.5}
\definecolor{purple}{rgb}{0.58,0,0.82}

\lstdefinelanguage{none}{}

\lstset{
	language={Python},
	inputpath={../},
	backgroundcolor=\color{white},
	commentstyle=\color{green},
	keywordstyle=\color{blue},
	numberstyle=\color{gray}\scriptsize\ttfamily,
	stringstyle=\color{purple},
	basicstyle=\lst@ifdisplaystyle\footnotesize\fi\ttfamily,
	breakatwhitespace=false,
	breaklines=true,
	captionpos=b,
	keepspaces=true,
	numbers=left,
	numbersep=5pt,
	showspaces=false,
	showstringspaces=false,
	showtabs=false,
	tabsize=4,
	frame=single,
	morekeywords={},
	deletekeywords={},
	extendedchars=true,
	columns=fullflexible,
	inputencoding=utf8/cp1251,
	literate=%
		{~}{{\raise.25ex\hbox{$\mathtt{\sim}$}}}{1}
}

\titleformat*{\section}{\large\bfseries} 
\titleformat*{\subsection}{\normalsize\bfseries} 
\titleformat*{\subsubsection}{\normalsize\bfseries} 
\titleformat*{\paragraph}{\normalsize\bfseries} 
\titleformat*{\subparagraph}{\normalsize\bfseries} 

\counterwithin{figure}{section}
\counterwithin{equation}{section}
\counterwithin{table}{section}
\newcommand{\sign}[1][5cm]{\makebox[#1]{\hrulefill}}
\newcommand{\equipollence}{\quad\Leftrightarrow\quad}
\newcommand{\no}[1]{\overline{#1}}
\newcommand{\code}[1]{\lstinline[language=none]|#1|}
\newcommand{\data}[2]{
\paragraph{Описание.} #2.
\begin{figure}[H]
	\centering
	\includegraphics[width=\linewidth]{#1}
	\caption{Результат работы на данных #1.}
\end{figure}
}

\graphicspath{{../pics/}}
\captionsetup{justification=centering,margin=1cm}
\def\arraystretch{1.3}
\setlength\parindent{5ex}
\titlelabel{\thetitle.\quad}

\setitemize{topsep=0em, itemsep=0em}
\setenumerate{topsep=0em, itemsep=0em}


\begin{document}

\begin{titlepage}
\begin{center}
	\textbf{Санкт-Петербургский Политехнический Университет \\Петра Великого}\\[0.3cm]
	\small Институт компьютерных наук и технологий \\[0.3cm]
	\small Кафедра компьютерных систем и программных технологий\\[4cm]
	
	\textbf{ОТЧЕТ}\\ \textbf{по расчетному заданию}\\[0.5cm]
	\textbf{<<Линейное программирование>>}\\[0.1cm]
	\textbf{Системный анализ и принятие решений}\\[8.0cm]
\end{center}

\begin{flushright}
	\begin{minipage}{0.4\textwidth}
		\begin{flushleft}
			\small \textbf{Работу выполнил студент}\\[3mm]
			\small группа 33501/4 \hspace*{6mm} Дьячков В.В.\\[5mm]
			
			\small \textbf{Преподаватель}\\[5mm]
		 	\small \sign[3cm] \hspace*{5mm} Сабонис С.С.\\[0.5cm]
		\end{flushleft}
	\end{minipage}
\end{flushright}

\vfill

\begin{center}
	\small Санкт-Петербург\\
	\small \today
\end{center}
\end{titlepage}

\addtocounter{page}{1}

\tableofcontents
%\newpage
\listoffigures
%\listoftables
\newpage

\section{Техническое задание}

\begin{enumerate}
	\setlength{\itemsep}{0em}
	\item Решить задачу методом Лагранжа при заданном ограничении;
	\item Решить задачу методом Била при заданных ограничениях;
	\item Решить задачу методом проекции градиента при заданных ограничениях;
	\item Решить задачу методом штрафных функций или методом барьерных функций при
заданном ограничении;
	\item Решить задачу методом возможных направлений при заданном ограничении.
\end{enumerate}

\section{Исходные данные}

\paragraph{Вариант 32}

Дана задача нелинейного программирования:
\begin{equation*}
	\max f(X) = \max \left( -31 x_1^2 - 34 x_2^2 + 4 x_1 x_2 + 286 x_1 + 388 x_2 \right)
\end{equation*}
Заданы коэффициенты $a_{ij}$:
\begin{center}
\begin{multicols}{5}
	$a_{11} = 7$\\
	$a_{21} = 10$\\
	$a_{31} = -1$\\
	$a_{41} = 0$\\
	$a_{51} = 0$\\
\end{multicols}
\begin{multicols}{5}
	$a_{12} = 12$\\
	$a_{22} = 8$\\
	$a_{32} = 0$\\
	$a_{42} = -1$\\
	$a_{52} = 1$\\
\end{multicols}
\end{center}
Заданы коэффициенты $b_i$:
\begin{multicols}{6}
	\centering
	$b_1 = 84$\\
	$b_2 = 80$\\
	$b_3 = 0$\\
	$b_4 = 0$\\
	$b_5 = 5$\\
	$b_6 = 400$\\
\end{multicols}
\noindent Заданы коэффициенты $d_i$:
\begin{multicols}{2}
	\centering
	$d_1 = 16$\\
	$d_2 = 25$\\
\end{multicols}

\section{Решение методом Лагранжа}

Решим задачу при ограничении:
\begin{equation*}
a_{51} x_1 + a_{52} x_2 = b_5 
\Longleftrightarrow
x_2 = 5
\Longleftrightarrow
x_2 - 5 = 0
\end{equation*}

В методе Лагранжа исходная задача условной оптимизации сводится к задаче безусловной оптимизации -- задаче поиска стационарной точки функции Лагранжа, являющийся точкой локального максимума функции $L(X, V)$ по аргументу $X$. Запишем функцию Лагранжа:
\begin{equation*}
L(X, V) = -31 x_1^2 - 34 x_2^2 + 4 x_1 x_2 + 286 x_1 + 388 x_2 + V_1 \left(x_2 - 5\right)
\end{equation*}

Сформулируем условие стационарности:
\begin{equation*}
\begin{cases}
\dfrac{\partial L}{\partial x_1} = -62 x_1 + 4 x_2 + 286 = 0 \\
\dfrac{\partial L}{\partial x_2} = -68 x_2 + 4 x_1 + 388 + V_1 = 0 \\
\dfrac{\partial L}{\partial V_1} = x_2 - 5 = 0
\end{cases}
\end{equation*}

Решая систему уравнений получим:
\begin{equation*}
\begin{cases}
x_2 = 5 \\
-62 x_1 + 20 + 286 = 0 \\
-340 + 4 x_1 + 388 + V_1 = 0
\end{cases}
\Rightarrow
\begin{cases}
x_2 = 5 \\
x_1 \approx \dfrac{286}{62} \approx 4.94 \\
V_1 \approx -67.76
\end{cases}
\end{equation*}

Определим матрицу Гессе $H_L(X, V)$ и убедимся в ее отрицательной определенности:
\begin{equation*}
H_L(X, V) =
\begin{pmatrix}
	-62 & 4 \\
	4 & -68
\end{pmatrix} = H_L 
\end{equation*}

\newtheorem*{theorem1}{Критерий отрицательной определенности квадратичной формы}
\begin{theorem1}
Для отрицательной определенности квадратичной формы необходимо и достаточно, чтобы угловые миноры четного порядка ее матрицы были положительны, а нечетного порядка — отрицательны.
\end{theorem1}

Найдем главные миноры матрицы $H_L$:
\begin{multicols}{2}
\centering
$\Delta_1 = \begin{vmatrix} -62 \end{vmatrix} = -62$\\
$\Delta_2 = \begin{vmatrix}
	-62 & 4 \\
	4 & -68
\end{vmatrix} = 4200$
\end{multicols}

По критерию отрицательной определенности квадратичной формы, матрица $H$ отрицательно определена. Таким образом, в соответствии с условиями второго порядка точка $(X^*, V^*)$ является точкой максимума $L(X, V)$ по $X$, а точка $X^* \approx (4.94, 5)$ -- решением задачи условной оптимизации, $f(X^*) \approx 1845$. 

\section{Необходимые условия оптимальности при линейных ограничениях}

Запишем необходимые условия оптимальности для задачи при ограничениях:
\begin{equation*}
\begin{cases}
a_{11} x_1 + a_{12} x_2 \leq b_1 \\
a_{21} x_1 + a_{22} x_2 \leq b_2 \\
a_{31} x_1 + a_{32} x_2 \leq b_3 \\
a_{41} x_1 + a_{42} x_2 \leq b_4 \\
\end{cases}
\end{equation*}

\section{Решение методом Била}

Запишем ограничения в канонической форме:
\begin{equation*}
\begin{cases}
a_{11} x_1 + a_{12} x_2 \leq b_1 \\
a_{21} x_1 + a_{22} x_2 \leq b_2 \\
a_{31} x_1 + a_{32} x_2 \leq b_3 \\
a_{41} x_1 + a_{42} x_2 \leq b_4 \\
\end{cases}
\Longleftrightarrow
\begin{cases}
7 x_1 + 12 x_2 \leq 84 \\
10 x_1 + 8 x_2 \leq 80 \\
-x_1 \leq 0 \\
-x_2 \leq 0 \\
\end{cases}
\Longleftrightarrow
\begin{cases}
7 x_1 + 12 x_2 + x_3 = 84 \\
10 x_1 + 8 x_2 + x_4 = 80 \\
x_i \geq 0, i = \overline{1,4}
\end{cases}
%\Longleftrightarrow
%\begin{pmatrix}
%	7 & 12 \\
%	10 & 8 \\
%	-1 & 0 \\
%	0 & -1
%\end{pmatrix}
%\begin{pmatrix}
%	x_1 \\
%	x_2
%\end{pmatrix} \leq
%\begin{pmatrix}
%	84 \\
%	80 \\
%	0 \\
%	0
%\end{pmatrix}
\end{equation*}

Запишем частные производные:

\begin{equation*}
\begin{cases}
\dfrac{\partial f}{\partial x_1} = -62 x_1 + 4 x_2 + 286 \\
\dfrac{\partial f}{\partial x_2} = -68 x_2 + 4 x_1 + 388
\end{cases}
\end{equation*}

Пусть $\text{Б}_0 = \begin{pmatrix} 3 & 4 \end{pmatrix}$, тогда $X^{(0)} = \begin{pmatrix} 0 & 0 & 84 & 80 \end{pmatrix}^T$. Заполним симплекс-таблицу для опорной точки $X^{(0)}$, записав в последнюю строку формулы для частных производных:

\begin{table}[H]
\begin{center}
	\def\tabcolsep{15pt}
	\def\arraystretch{1.3}
	\caption{Базис $x_3, x_4$}
	\label{tab:simplex_1}
	\begin{tabular}{|c||c|c||c|}
		\hline
		$X^{(0)}$ & $x_1$ & $x_2$ & $b$ \\ 
		\hhline{|=#==#=|}
		$x_3$ & $-7$ & $-12$ & $84$ \\ 
		\hline
		$x_4$ & $-10$ & $-8$ & $80$ \\ 
		\hhline{|=#==#=|}
		$\partial f / \partial x_j$ & $-62 x_1 + 4 x_2 + 286$ & $-68 x_2 + 4 x_1 + 388$ &  \\ 
		\hline
	\end{tabular}
\end{center}
\end{table}

Базис является допустимым, так как $b \geq 0$, но не является допустимым, так как $c = \begin{pmatrix} 286 & 388 \end{pmatrix} \nleqslant 0$.

Выберем как разрешающий столбец переменную $x_k	$, соответствующую максимальному значению производной целевой функции в точке $X^{(0)}$: 

$k = \argmax\limits_i\left\{\dfrac{\partial f}{\partial x_i}\left(X^{(0)}\right)\right\} = 2 \Rightarrow x_2$.

Оценим ситуацию в опорной точке $X^{(0)}$. Найдем соотношения между приращениями свободной переменной $x_2$ и изменениями базисных переменных $x_3$ и $x_4$ и частной производной по $x_2$:
\begin{itemize}
\item $\dfrac{\partial f}{\partial x_2}\left(X^{(0)}\right) = 0$ при $x_2 \approx 5.7$
\item $x_3 = 0$ при $x_2 = 7$
\item $x_4 = 0$ при $x_2 = 10$
\end{itemize}

Производная обращается в ноль раньше базисных переменных, поэтому введем в задачу новую свободную переменную:
\begin{equation*}
u_1 = \frac{\partial f}{\partial x_2} = -68 x_2 + 4 x_1 + 388
\Longleftrightarrow
x_2 = \frac{x_1}{17} - \frac{u_1}{68} + \frac{97}{17}
\end{equation*}

Заполним промежуточную таблицу, содержащую строку для новой переменной $u_1$. На пересечении $u_1$ и $x_2$ находится разрешающий элемент $-68$. 
\begin{table}[H]
\begin{center}
	\def\tabcolsep{15pt}
	\def\arraystretch{1.3}
	\caption{Промежуточная таблица}
	\label{tab:simplex_1}
	\begin{tabular}{|c||c|c||c|}
		\hline
		$--$ & $x_1$ & $x_2$ & $b$ \\ 
		\hhline{|=#==#=|}
		$u_1$ & $4$ & \textcolor{red}{\boldmath$-68$} & $388$ \\ 
		\hline
		$x_3$ & $-7$ & $-12$ & $84$ \\ 
		\hline
		$x_4$ & $-10$ & $-8$ & $80$ \\ 
		\hline
	\end{tabular}
\end{center}
\end{table}

Произведем перерасчет в соответствии с правилом перерасчета симплекс-таблиц.

\begin{table}[H]
\begin{center}
	\def\tabcolsep{15pt}
	\def\arraystretch{1.3}
	\caption{Промежуточная таблица}
	\label{tab:simplex_1}
	\begin{tabular}{|c||c|c||c|}
		\hline
		$X^{(1)}$ & $x_1$ & $u_1$ & $b$ \\ 
		\hhline{|=#==#=|}
		$x_2$ & $4$ & \textcolor{red}{\boldmath$1$} & $-388$ \\ 
		\hline
		$x_3$ & $524$ & $-12$ & $-1056$ \\ 
		\hline
		$x_4$ & $712$ & $-8$ & $-2336$ \\ 
		\hline
	\end{tabular}
\end{center}
\end{table}

Поделим на $-68$. Выразим целевую функцию через свободную переменную 
\begin{align*}
f(u_1, x_1) &= -31 x_1^2 - 34 \left(\frac{x_1}{17} - \frac{u_1}{68} + \frac{97}{17}\right)^2 + 4 x_1 \left(\frac{x_1}{17} - \frac{u_1}{68} + \frac{97}{17}\right) + \\ &+ 286 x_1 + 388 \left(\frac{x_1}{17} - \frac{u_1}{68} + \frac{97}{17}\right)
\end{align*}

\begin{table}[H]
\begin{center}
	\def\tabcolsep{15pt}
	\def\arraystretch{1.3}
	\caption{Базис $x_2$, $x_3$ и $x_4$}
	\label{tab:simplex_1}
	\begin{tabular}{|c||c|c||c|}
		\hline
		$X^{(1)}$ & $x_1$ & $u_1$ & $b$ \\ 
		\hhline{|=#==#=|}
		$x_2$ & $-\frac{1}{17}$ & \textcolor{red}{\boldmath$-\frac{1}{68}$} & $\frac{97}{17}$ \\ 
		\hline
		$x_3$ & $-\frac{131}{17}$ & $\frac{3}{17}$ & $\frac{264}{17}$ \\ 
		\hline
		$x_4$ & $-\frac{178}{17}$ & $\frac{2}{17}$ & $\frac{584}{17}$ \\ 
		\hhline{|=#==#=|}
		$\partial f / \partial x_j$ & $-62 x_1 + 4 x_2 + 286$ & $-68 x_2 + 4 x_1 + 388$ &  \\ 
		\hline
	\end{tabular}
\end{center}
\end{table}

\section{Решение методом проекции градиента}

Решим задачу при ограничениях:
\begin{equation*}
\begin{cases}
a_{11} x_1 + a_{12} x_2 \leq b_1 \\
a_{21} x_1 + a_{22} x_2 \leq b_2 \\
a_{31} x_1 + a_{32} x_2 \leq b_3 \\
a_{41} x_1 + a_{42} x_2 \leq b_4 \\
\end{cases}
\end{equation*}

\section{Необходимые условия оптимальности при квадратичных ограничениях}

Решим задачу при ограничении:
\begin{equation*}
d_1 x_1^2 + d_2 x_2^2 \leq b_6
\Longleftrightarrow
16 x_1^2 + 25 x_2^2 \leq 400
\end{equation*}

\section{Решение методом барьерных/штрафных функций}

Решим задачу при ограничении:
\begin{equation*}
d_1 x_1^2 + d_2 x_2^2 \leq b_6
\end{equation*}

\section{Решение методом возможных направлений}

Решим задачу при ограничении:
\begin{equation*}
d_1 x_1^2 + d_2 x_2^2 \leq b_6
\end{equation*}

\end{document}