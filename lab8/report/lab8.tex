\documentclass[a4paper,14pt]{extarticle}

\usepackage[utf8x]{inputenc}
\usepackage[T1,T2A]{fontenc}
\usepackage[russian]{babel}
\usepackage{hyperref}
\usepackage{indentfirst}
\usepackage{here}
\usepackage{array}
\usepackage{graphicx}
\usepackage{caption}
\usepackage{subcaption}
\usepackage{chngcntr}
\usepackage{amsmath}
\usepackage{amssymb}
\usepackage{amsthm}
\usepackage{pgfplots}
\usepackage{pgfplotstable}
\usepackage[left=2cm,right=2cm,top=2cm,bottom=2cm,bindingoffset=0cm]{geometry}
\usepackage{multicol}
\usepackage{askmaps}
\usepackage{titlesec}
\usepackage{listings}
\usepackage{color}
\usepackage{enumerate}
\usepackage{hhline}
\usepackage{enumitem}
\usepackage{courier}
\usepackage{wrapfig}
\usetikzlibrary{arrows,automata}

\setitemize{itemsep=0em}
\setenumerate{itemsep=0em}

\theoremstyle{definition}

\pgfkeys{/pgf/number format/.cd,1000 sep={\,}}

\definecolor{green}{rgb}{0,0.6,0}
\definecolor{gray}{rgb}{0.5,0.5,0.5}
\definecolor{purple}{rgb}{0.58,0,0.82}

\lstset{
	language=python,
	backgroundcolor=\color{white},   
	commentstyle=\color{green},
	keywordstyle=\color{blue},
	numberstyle=\tiny\color{gray},
	stringstyle=\color{purple},
	basicstyle=\footnotesize\ttfamily,
	breakatwhitespace=false,
	breaklines=true,
	captionpos=b,
	keepspaces=true,
	numbers=left,
	numbersep=5pt,
	showspaces=false,
	showstringspaces=false,
	showtabs=false,
	tabsize=2,
	frame=single,
	inputpath={../code/}
}

\renewcommand{\le}{\ensuremath{\leqslant}}
\renewcommand{\leq}{\ensuremath{\leqslant}}
\renewcommand{\ge}{\ensuremath{\geqslant}}
\renewcommand{\geq}{\ensuremath{\geqslant}}
\renewcommand{\epsilon}{\ensuremath{\varepsilon}}
\renewcommand{\phi}{\ensuremath{\varphi}}
\renewcommand{\thefigure}{\arabic{figure}} 	
\newcommand{\norm}[1]{\left\lVert#1\right\rVert}
\newcommand*\sfrac[2]{{}^{#1}\!/_{#2}}

%\titleformat*{\section}{\large\bfseries} 
\titleformat*{\subsection}{\normalsize\bfseries} 
\titleformat*{\subsubsection}{\normalsize\bfseries} 
\titleformat*{\paragraph}{\normalsize\bfseries} 
\titleformat*{\subparagraph}{\normalsize\bfseries} 

\counterwithin{figure}{section}
\counterwithin{equation}{section}
\counterwithin{table}{section}
\newcommand{\sign}[1][5cm]{\makebox[#1]{\hrulefill}}
\graphicspath{{../pics/}}
\captionsetup{justification=centering,margin=1cm}
\setlength\parindent{5ex}
\def\arraystretch{1.3}
\def\tabcolsep{12pt}
%\titlelabel{\thetitle.\quad}

\DeclareMathOperator*{\argmin}{argmin}
\DeclareMathOperator*{\argmax}{argmax}

\begin{document}

\begin{titlepage}
\begin{center}
	\textbf{Санкт-Петербургский Политехнический Университет \\Петра Великого}\\[0.3cm]
	\small Институт компьютерных наук и технологий \\[0.3cm]
	\small Кафедра компьютерных систем и программных технологий\\[4cm]
	
	\textbf{ОТЧЕТ}\\ \textbf{по расчетному заданию}\\[0.5cm]
	\textbf{<<Построение моделей>>}\\[0.1cm]
	\textbf{Системный анализ и принятие решений}\\[8.0cm]
\end{center}

\begin{flushright}
	\begin{minipage}{0.4\textwidth}
		\begin{flushleft}
			\small \textbf{Работу выполнил студент}\\[3mm]
			\small группа 33501/4 \hspace*{6mm} Дьячков В.В.\\[5mm]
			
			\small \textbf{Преподаватель}\\[5mm]
		 	\small \sign[3cm] \hspace*{5mm} Сабонис С.С.\\[0.5cm]
		\end{flushleft}
	\end{minipage}
\end{flushright}

\vfill

\begin{center}
	\small Санкт-Петербург\\
	\small \the\year
\end{center}
\end{titlepage}

\addtocounter{page}{1}

\tableofcontents
\listoftables
\listoffigures
\newpage

\section{Техническое задание}

\paragraph{Вариант 32.} 
Рассматривается работа столовой самообслуживания. Обеды выдают $K$ поваров. Среднее время выдачи обеда на одного посетителя равно $t_1$ минут. Плотность потока посетителей около $N$ человек в минуту. В очереди могут одновременно стоять не более $m$ человек. В среднем посетитель стоит в очереди $t_2$ минут, после чего покидает столовую. Если посетителя начали обслуживать, то обслуживание не прерывается. На обед посетитель в среднем затрачивает $t_3$ минут. 

Определить, сколько времени потратит посетитель в столовой, если количество мест за столами всегда достаточно для размещения лиц, уже получивших обед. Определить среднее число занятых поваров и среднее число ожидающих посетителей. Определить вероятности того, что посетитель:
\begin{itemize}
	\item успешно пообедает;
	\item уйдет, не дождавшись своей очереди;
	\item уйдет, не имея возможности встать в очередь;
\end{itemize} 

\section{Исходные данные}

\begin{table}[H]
	\begin{center}
		\caption{Исходные данные}
		\def\tabcolsep{12pt}
		\begin{tabular}{|c|c|c|c|c|c|}
			\hline
			$N$, чел./мин. & $t_1$, мин & $t_2$, мин & $t_3$, мин & $K$, чел. & $m$, чел. \\
			\hline
			$1$ & $4$ & $10$ & $10$ & $3$ & $16$ \\
			\hline	
		\end{tabular}
	\end{center}
\end{table}

\section{Система массового обслуживания}

На рис. \ref{fig:structure} изображена структура сети $M/M/3/16$.
\begin{figure}[H]
	\centering
	\begin{tikzpicture}[start chain=going right,>=latex,node distance=0pt]
	\node[rectangle split, minimum height=2cm,draw, rectangle split horizontal,on chain] (wa) {
		$\max=16$
		\nodepart{two} $\circ$
		\nodepart{three} $\circ$
		\nodepart{four} $\circ$
	};
	\draw[<-] (wa.west) -- +(-1.2,0) node[left] {$\lambda$};
	\draw[->] (wa.north) -- +(45:1.2) coordinate (B0);
	\node[align=center,left] at (B0) {$\nu$};
	\draw[->] (wa.east) -- +(45:1.8) coordinate (B1);
	\draw[->] (wa.east) -- +(0:1.28) coordinate (B2);
	\draw[->] (wa.east) -- +(-45:1.8) coordinate (B3);
	\node[draw,rectangle,on chain,minimum size=1cm] at (B1) (se1) {$\mu$};
	\node[draw,rectangle,on chain,minimum size=1cm] at (B2) (se2) {$\mu$};
	\node[draw,rectangle,on chain,minimum size=1cm] at (B3) (se3) {$\mu$};
	\draw[-] (se1.east) --+(-45:1.8) coordinate (C1);
	\draw[-] (se2.east) --+(0:1.2) coordinate (C2);
	\draw[-] (se3.east) --+(45:1.8) coordinate (C3);
	\draw[->] (C2)--+(0:0.5)node[right] (C4) {};
	\node[draw,rectangle,on chain,minimum size=2cm] at (C4.west) {$+T$};
	\end{tikzpicture}
	\caption{Структура сети}
	\label{fig:structure}
\end{figure}

На рис. \ref{fig:state_graph} изображен граф состояний рассматриваемой сети.
\begin{figure}[H]
	\centering
	\begin{tikzpicture}[->,>=stealth',shorten >=1pt,auto,node distance=2.4cm,semithick]
		\tikzstyle{every state}=[fill=white,draw=black,text=black]
		
		\node[state] (0) {$0$};
		\node[state] (1) [right of=0] {$1$};
		\node[state] (2) [right of=1] {$2$};
		\node[state] (3) [right of=2] {$3$};
		\node[state] (4) [right of=3] {$4$};
		\node[state,draw=white] (5) [right of=4] {\Large$\dots$};
		\node[state] (100) [right of=5] {$19$};
		
		\path 
		(0) edge [bend left] node {$\lambda$} (1)
		(1) edge [bend left] node {$\mu$} (0)
		(1) edge [bend left] node {$\lambda$} (2)
		(2) edge [bend left] node {$2\mu$} (1)
		(2) edge [bend left] node {$\lambda$} (3)
		(3) edge [bend left] node {$3\mu$} (2)
		(3) edge [bend left] node {$\lambda$} (4)
		(4) edge [bend left] node {$3\mu + \nu$} (3)
		(4) edge [bend left] node {$\lambda$} (5)
		(5) edge [bend left] node {$3\mu + 2\nu$} (4)
		(5) edge [bend left] node {$\lambda$} (100)
		(100) edge [bend left] node {$3\mu + 16\nu$} (5);
	\end{tikzpicture}
	\caption{Граф состояний}
	\label{fig:state_graph}
\end{figure}

Граф гибели и размножения, следовательно можно применить формулу \ref{eq:p} для расчета вероятности нахождения системы в состоянии $i$:
\begin{equation}
\label{eq:p}
P_i = \begin{cases}
\left( 1 + \sum \limits_{i=1}^{N} \prod \limits_{j=1}^{i} \frac{\lambda_j}{\mu_j} \right)^{-1} , &\text{если } i = 0 \\
\prod \limits_{j=1}^{i} \frac{\lambda_j}{\mu_j}, &\text{если } i \neq 0
\end{cases}
\end{equation}

В таблице \ref{tab:data} приведена вероятность каждого из состояний системы, количество заявок, находящихся в очереди, а так же число занятых каналов.
\begin{table}[H]
	\centering
	\def\arraystretch{1.1}
	\def\tabcolsep{25pt}
	\pgfplotstabletypeset[col sep=comma,
		columns={state, prob, busy, queue},
		column type/.add={|c|}{},
		columns/state/.style={fixed, column name={$i$}},
		columns/prob/.style={fixed, precision=5, zerofill, column name={$P_i$}},
		columns/queue/.style={fixed, column name={$\overline{n_\text{о}}_i$}},
		columns/busy/.style={fixed, column name={$\overline{k_\text{з}}_i$}},
		every nth row={1}{before row=\hline},
		every head row/.style={before row=\hline, after row=\hline, },
		every last row/.style={after row=\hline}
	]{../data/data.csv}
	\caption{Вероятности состояний системы}
	\label{tab:data}
\end{table}

Из найденных вероятностей найдем характеристики системы массового обслуживания:
\begin{itemize}
	\item Сумма вероятностей $\sum\limits_{i=0}^{N} P_i = 1$, значит вероятности найдены верно.
	
	\item Среднее количество занятых каналов найдем как сумму поэлементных произведений вероятностей состояний на количество занятых каналов в этих состояних: 
	\begin{displaymath}
		\overline{k_\text{з}} = \sum \limits_{i=0}^{N} \overline{k_\text{з}}_i \approx 2.80873
	\end{displaymath}
	
	\item Коэффициент загрузки каналов найдем как отношение среднего количества занятых каналов к общему количеству каналов в системе:
	\begin{displaymath}
		\eta = \dfrac{\overline{k_\text{з}}}{k} \approx 0.93624
	\end{displaymath}
	
	\item Среднюю длину очереди найдем как сумму поэлементных произведений вероятностей состояний на длину очереди в этих состояних: 
	\begin{displaymath}
		\overline{n_\text{о}} = \sum \limits_{i=0}^{N} \overline{n_\text{о}}_i \approx 2.97698
	\end{displaymath}
	
	\item Вероятность отказа из-за заполненной очереди равна вероятности нахождения системы в последнем состоянии: 
	\begin{displaymath}
		P_\text{отк} \approx 0.00012
	\end{displaymath}
	
	\item Вероятность ухода из очереди равна отношению интенсивности ухода к интенсивности поступления заявок, умноженному на среднюю длину очереди: 
	\begin{displaymath}
		P_\text{ух} \approx 0.29770
	\end{displaymath}
	
	\item Вероятность успешного обслуживания найдем как произведение вероятностей отстутствия отказа обслуживания и ухода заявки из очереди: 
	\begin{displaymath}
		P_\text{усп} = (1 - P_\text{отк}) \cdot (1 - P_\text{ух}) \approx 0.70222
	\end{displaymath}
	
	\item Среднее количество человек в системе найдем как сумму средней длины очереди и среднего количества занятых каналов: 
	\begin{displaymath}
		j \approx 5.78571
	\end{displaymath}
	
	\item Среднее время нахождения в системе найдем с помощью закону Литтла, прибавив время обеда, умноженное на вероятность успешного обслуживания:
	\begin{displaymath}
		t_c = \dfrac{j}{\lambda} + T = \dfrac{j}{\lambda} + P_\text{усп} \cdot t_3 \approx 12.80789
	\end{displaymath}
\end{itemize}
	
\end{document}