\documentclass[a4paper,14pt]{extarticle}

\usepackage[utf8x]{inputenc}
\usepackage[T1,T2A]{fontenc}
\usepackage[russian]{babel}
\usepackage{hyperref}
\usepackage{indentfirst}
\usepackage{here}
\usepackage{array}
\usepackage{graphicx}
\usepackage{caption}
\usepackage{subcaption}
\usepackage{chngcntr}
\usepackage{amsmath}
\usepackage{amssymb}
\usepackage{amsthm}
\usepackage{pgfplots}
\usepackage{pgfplotstable}
\usepackage[left=2cm,right=2cm,top=2cm,bottom=2cm,bindingoffset=0cm]{geometry}
\usepackage{multicol}
\usepackage{askmaps}
\usepackage{titlesec}
\usepackage{listings}
\usepackage{color}
\usepackage{enumerate}
\usepackage{hhline}
\usepackage{enumitem}
\usepackage{courier}
\usepackage{wrapfig}
\usetikzlibrary{arrows,automata}

\setitemize{itemsep=0em}
\setenumerate{itemsep=0em}

\theoremstyle{definition}

\pgfkeys{/pgf/number format/.cd,1000 sep={\,}}

\definecolor{green}{rgb}{0,0.6,0}
\definecolor{gray}{rgb}{0.5,0.5,0.5}
\definecolor{purple}{rgb}{0.58,0,0.82}

\lstset{
	language=python,
	backgroundcolor=\color{white},   
	commentstyle=\color{green},
	keywordstyle=\color{blue},
	numberstyle=\tiny\color{gray},
	stringstyle=\color{purple},
	basicstyle=\footnotesize\ttfamily,
	breakatwhitespace=false,
	breaklines=true,
	captionpos=b,
	keepspaces=true,
	numbers=left,
	numbersep=5pt,
	showspaces=false,
	showstringspaces=false,
	showtabs=false,
	tabsize=2,
	frame=single,
	inputpath={../code/}
}

\renewcommand{\le}{\ensuremath{\leqslant}}
\renewcommand{\leq}{\ensuremath{\leqslant}}
\renewcommand{\ge}{\ensuremath{\geqslant}}
\renewcommand{\geq}{\ensuremath{\geqslant}}
\renewcommand{\epsilon}{\ensuremath{\varepsilon}}
\renewcommand{\phi}{\ensuremath{\varphi}}
\renewcommand{\thefigure}{\arabic{figure}} 	
\newcommand{\norm}[1]{\left\lVert#1\right\rVert}
\newcommand*\sfrac[2]{{}^{#1}\!/_{#2}}

%\titleformat*{\section}{\large\bfseries} 
\titleformat*{\subsection}{\normalsize\bfseries} 
\titleformat*{\subsubsection}{\normalsize\bfseries} 
\titleformat*{\paragraph}{\normalsize\bfseries} 
\titleformat*{\subparagraph}{\normalsize\bfseries} 

\counterwithin{figure}{section}
\counterwithin{equation}{section}
\counterwithin{table}{section}
\newcommand{\sign}[1][5cm]{\makebox[#1]{\hrulefill}}
\graphicspath{{../pics/}}
\captionsetup{justification=centering,margin=1cm}
\setlength\parindent{5ex}
\def\arraystretch{1.3}
\def\tabcolsep{12pt}
%\titlelabel{\thetitle.\quad}

\DeclareMathOperator*{\argmin}{argmin}
\DeclareMathOperator*{\argmax}{argmax}

\begin{document}

\begin{titlepage}
\begin{center}
	\textbf{Санкт-Петербургский Политехнический Университет \\Петра Великого}\\[0.3cm]
	\small Институт компьютерных наук и технологий \\[0.3cm]
	\small Кафедра компьютерных систем и программных технологий\\[4cm]
	
	\textbf{ОТЧЕТ}\\ \textbf{по расчетному заданию}\\[0.5cm]
	\textbf{<<Построение моделей>>}\\[0.1cm]
	\textbf{Системный анализ и принятие решений}\\[8.0cm]
\end{center}

\begin{flushright}
	\begin{minipage}{0.4\textwidth}
		\begin{flushleft}
			\small \textbf{Работу выполнил студент}\\[3mm]
			\small группа 33501/4 \hspace*{6mm} Дьячков В.В.\\[5mm]
			
			\small \textbf{Преподаватель}\\[5mm]
		 	\small \sign[3cm] \hspace*{5mm} Сабонис С.С.\\[0.5cm]
		\end{flushleft}
	\end{minipage}
\end{flushright}

\vfill

\begin{center}
	\small Санкт-Петербург\\
	\small \the\year
\end{center}
\end{titlepage}

\addtocounter{page}{1}

\tableofcontents
%\newpage
\listoffigures
\listoftables
\newpage

\section{Техническое задание}

\begin{enumerate}
	\item Упорядочить вершины графа -- провести разбиение множества вершин графа на непересекающиеся подмножества (уровни), показать уровни на графе;
	\item Определить наименьший путь (пути) на графе методом динамического программирования, выделить его на графе;
	\item Определить наибольший путь (пути) на графе методом динамического программирования, выделить его на графе. 
\end{enumerate}

\section{Исходные данные}

\paragraph{Вариант 32}

\begin{figure}[H]
\begin{center}
	\begin{tikzpicture}[->,>=stealth',shorten >=1pt,auto,node distance=2.8cm,
	                    semithick, inner sep=5pt]
	  \tikzstyle{every state}=[fill=white,draw=black,text=black]
	
	  \node[initial, state] (1) 					{$1$};
	  \node[state] 			(2) [right of=1] 		{$2$};
	  \node[state] 			(3) [above right of=2] 	{$3$};
	  \node[state] 			(4) [below right of=2] 	{$4$};
	  \node[state] 			(5) [below right of=3] 	{$5$};
	  \node[state] 			(6) [below right of=5] 	{$6$};
	  \node[state] 			(7) [below right of=4] 	{$7$};
	  \node[state] 			(8) [above right of=6] 	{$8$};
	  \node[state] 			(9) [right of=8] 		{$9$};
	
	  \path (1)	edge 				node 		{6} (2)
	  			edge [bend left] 	node 		{7} (3)
	  			edge [bend right] 	node [swap] {4} (4)
	  		(2) edge 				node 		{6} (3)
	  			edge 				node 		{4} (4)
	  		(3) edge 				node 		{5} (5)
	  			edge [bend left] 	node 		{5} (8)
	  		(4) edge 				node 		{6} (5)
	  			edge 				node [swap] {4} (6)
	  			edge [bend right] 	node [swap] {6} (7)
	  		(5) edge 				node		{4} (6)
	  			edge 				node 		{3} (8)
	  		(6) edge 				node 		{5} (8)
	  		(7) edge [bend right] 	node [swap] {7} (8)
	  		(8) edge 				node 		{4} (9);
	\end{tikzpicture}
	\caption{Исходный граф}
	\label{fig:graph}
\end{center}
\end{figure}

\section{Разбиение множества вершин графа на уровни}

\begin{figure}[H]
\begin{center}
	\begin{tikzpicture}[->,>=stealth',shorten >=1pt,auto,node distance=2.8cm,
	                    semithick, inner sep=5pt]
		\tikzstyle{every state}=[fill=white,draw=black,text=black]
		\tikzstyle{every node}=[draw=none]	
	
		\node[initial, state] (1) 					{$1$};
		\node[state] 			(2) [right of=1] 		{$2$};
		\node[state] 			(3) [above right of=2] 	{$3$};
		\node[state] 			(4) [below right of=2] 	{$4$};
		\node[state] 			(5) [below right of=3] 	{$5$};
		\node[state] 			(6) [below right of=5] 	{$6$};
		\node[state] 			(7) [below right of=4] 	{$7$};
		\node[state] 			(8) [above right of=6] 	{$8$};
		\node[state] 			(9) [right of=8] 		{$9$};

		\path 	(1)	edge 				node 		{6} (2)
	  				edge [bend left] 	node 		{7} (3)
	  				edge [bend right] 	node [swap] {4} (4)
	  			(2) edge 				node 		{6} (3)
	  				edge 				node 		{4} (4)
	  			(3) edge 				node 		{5} (5)
	  				edge [bend left] 	node 		{5} (8)
	  			(4) edge 				node 		{6} (5)
	  				edge 				node [swap] {4} (6)
	  				edge [bend right] 	node [swap] {6} (7)
	  			(5) edge 				node		{4} (6)
	  				edge 				node 		{3} (8)
	  			(6) edge 				node 		{5} (8)
	  			(7) edge [bend right] 	node [swap] {7} (8)
	  			(8) edge 				node 		{4} (9);
	  		
		\draw [-,dashed,red,thick] (1.7,-4.5) -- (1.7,3.6);
		\draw [-,dashed,red,thick] (3.7,-4.5) -- (3.7,3.6);
		\draw [-,dashed,red,thick] (5.7,-4.5) -- (5.7,3.6);
		\draw [-,dashed,red,thick] (7.7,-4.5) -- (7.7,3.6);
		\draw [-,dashed,red,thick] (9.7,-4.5) -- (9.7,3.6);
		\draw [-,dashed,red,thick] (11.7,-4.5) -- (11.7,3.6);
		
		
		\node[draw, above of=1] at (0.7, 0.5) {$A_1$};
		\node[draw, above of=2] at (2.7, 0.5) {$A_2$};
		\node[draw, above of=3] at (4.7, 0.5) {$A_3$};
		\node[draw, above of=5] at (6.7, 0.5) {$A_4$};
		\node[draw, above of=6] at (8.7, 0.5) {$A_5$};
		\node[draw, above of=8] at (10.7, 0.5) {$A_6$};
		\node[draw, above of=9] at (12.7, 0.5) {$A_7$};
	\end{tikzpicture}
	\caption{Разбиение вершин на уровни}
	\label{fig:split}
\end{center}
\end{figure}

\section{Определение наименьшего пути}

Будем обозначать $l_i$ путь из $i$-ой вершины в 9-ую.

\begin{enumerate}
	\item[$A_7$\ ] $l_9 = 0 = l_9$
	\item[$A_6$\ ] $l_8 = 4 = l_8$
	\item[$A_5$\ ] $l_6 = 5 + l_8 = 5 + 4 = 9 = l_6$
	\item[$A_4$\ ] $l_5 = \mintwo{3 + l_8}{4 + l_6} = \mintwo{3 + 4}{4 + 9} = 7$\\
	$l_7 = 7 + l_8 = 7 + 4 = 11$
	\item[$A_3$\ ] $l_3 = \mintwo{5 + l_8}{5 + l_5} = \mintwo{5 + 4}{5 + 7} = 9$
\end{enumerate}

\section{Определение наибольшего пути}

\end{document}