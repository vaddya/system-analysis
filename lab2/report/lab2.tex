\documentclass[a4paper,14pt]{extarticle}

\usepackage[utf8x]{inputenc}
\usepackage[T1]{fontenc}
\usepackage[russian]{babel}
\usepackage{hyperref}
\usepackage{indentfirst}
\usepackage{here}
\usepackage{array}
\usepackage{graphicx}
\usepackage{caption}
\usepackage{subcaption}
\usepackage{chngcntr}
\usepackage{amsmath}
\usepackage{amssymb}
\usepackage[left=2cm,right=2cm,top=2cm,bottom=2cm,bindingoffset=0cm]{geometry}
\usepackage{multicol}
\usepackage{multirow}
\usepackage{titlesec}
\usepackage{listings}
\usepackage{listingsutf8}
\usepackage{color}
\usepackage{enumitem}
\usepackage{cmap}
\usepackage{url}

\definecolor{green}{rgb}{0,0.6,0}
\definecolor{gray}{rgb}{0.5,0.5,0.5}
\definecolor{purple}{rgb}{0.58,0,0.82}

\lstdefinelanguage{none}{}

\lstset{
	language={Python},
	inputpath={../},
	backgroundcolor=\color{white},
	commentstyle=\color{green},
	keywordstyle=\color{blue},
	numberstyle=\color{gray}\scriptsize\ttfamily,
	stringstyle=\color{purple},
	basicstyle=\lst@ifdisplaystyle\footnotesize\fi\ttfamily,
	breakatwhitespace=false,
	breaklines=true,
	captionpos=b,
	keepspaces=true,
	numbers=left,
	numbersep=5pt,
	showspaces=false,
	showstringspaces=false,
	showtabs=false,
	tabsize=4,
	frame=single,
	morekeywords={},
	deletekeywords={},
	extendedchars=true,
	columns=fullflexible,
	inputencoding=utf8/cp1251,
	literate=%
		{~}{{\raise.25ex\hbox{$\mathtt{\sim}$}}}{1}
}

\titleformat*{\section}{\large\bfseries} 
\titleformat*{\subsection}{\normalsize\bfseries} 
\titleformat*{\subsubsection}{\normalsize\bfseries} 
\titleformat*{\paragraph}{\normalsize\bfseries} 
\titleformat*{\subparagraph}{\normalsize\bfseries} 

\counterwithin{figure}{section}
\counterwithin{equation}{section}
\counterwithin{table}{section}
\newcommand{\sign}[1][5cm]{\makebox[#1]{\hrulefill}}
\newcommand{\equipollence}{\quad\Leftrightarrow\quad}
\newcommand{\no}[1]{\overline{#1}}
\newcommand{\code}[1]{\lstinline[language=none]|#1|}
\newcommand{\data}[2]{
\paragraph{Описание.} #2.
\begin{figure}[H]
	\centering
	\includegraphics[width=\linewidth]{#1}
	\caption{Результат работы на данных #1.}
\end{figure}
}

\graphicspath{{../pics/}}
\captionsetup{justification=centering,margin=1cm}
\def\arraystretch{1.3}
\setlength\parindent{5ex}
\titlelabel{\thetitle.\quad}

\setitemize{topsep=0em, itemsep=0em}
\setenumerate{topsep=0em, itemsep=0em}


\begin{document}

\begin{titlepage}
\begin{center}
	\textbf{Санкт-Петербургский Политехнический Университет \\Петра Великого}\\[0.3cm]
	\small Институт компьютерных наук и технологий \\[0.3cm]
	\small Кафедра компьютерных систем и программных технологий\\[4cm]
	
	\textbf{ОТЧЕТ}\\ \textbf{по расчетному заданию}\\[0.5cm]
	\textbf{<<Линейное программирование>>}\\[0.1cm]
	\textbf{Системный анализ и принятие решений}\\[8.0cm]
\end{center}

\begin{flushright}
	\begin{minipage}{0.4\textwidth}
		\begin{flushleft}
			\small \textbf{Работу выполнил студент}\\[3mm]
			\small группа 33501/4 \hspace*{6mm} Дьячков В.В.\\[5mm]
			
			\small \textbf{Преподаватель}\\[5mm]
		 	\small \sign[3cm] \hspace*{5mm} Сабонис С.С.\\[0.5cm]
		\end{flushleft}
	\end{minipage}
\end{flushright}

\vfill

\begin{center}
	\small Санкт-Петербург\\
	\small \today
\end{center}
\end{titlepage}

\addtocounter{page}{1}

\section{Техническое задание}

\begin{enumerate}
	\setlength{\itemsep}{0em}
	\item Привести задачу к канонической форме;
	\item Решить задачу геометрическим методом;
	\item Обозначить все опорные точки (в том числе недопустимые) и записать соответствующие им наборы базисных переменных, рассчитать значение целевой функции в каждой опорной точке (решить задачу методом полного перебора опорных точек);
	\item Решить задачу симплекс-методом в матричной форме;
	\item Решить задачу симплекс-методом в табличной форме;
	\item Ввести дополнительное ограничение, отсекающее оптимальную точку. Решить новую задачу двойственным симплекс-методом в табличной форме, в качестве начального базиса новой задачи использовать оптимальный базис исходной задачи;
	\item Сформулировать задачу, двойственную по отношению к исходной. 
\end{enumerate}

\section{Исходные данные}

\paragraph{Вариант 32}

Дана задача линейного программирования:
\begin{equation*}
\begin{cases}
	\max \left( 2 x_1 + 3 x_2 \right) \\
	x_1 + x_2 \le 4.8 \\
	-3 x_1 - x_2 \le -3.5 \\
	x_1 \ge 0 \\
	x_2 \ge 0
\end{cases}
\end{equation*}

\section{Приведение к канонической форме}

\begin{equation*}
\begin{cases}
	\max \left( 2 x_1 + 3 x_2 \right) \\
	x_1 + x_2 + x_3 = 4.8 \\
	-3 x_1 - x_2 + x_4 = -3.5 \\
	x_i \ge 0, i = \overline{1,4}
\end{cases}
\end{equation*}

\section{Решение геометрическим методом}

\section{Нахождение опорных точек}

\section{Решение симплекс-методом в матричной форме}

\section{Решение симплекс-методом в табличной форме}

\section{Введение дополнительного ограничения}

\section{Двойственная задача}

\end{document}