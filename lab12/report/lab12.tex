\documentclass[a4paper,14pt]{extarticle}

\usepackage[utf8x]{inputenc}
\usepackage[T1]{fontenc}
\usepackage[russian]{babel}
\usepackage{hyperref}
\usepackage{indentfirst}
\usepackage{here}
\usepackage{array}
\usepackage{graphicx}
\usepackage{caption}
\usepackage{subcaption}
\usepackage{chngcntr}
\usepackage{amsmath}
\usepackage{amssymb}
\usepackage[left=2cm,right=2cm,top=2cm,bottom=2cm,bindingoffset=0cm]{geometry}
\usepackage{multicol}
\usepackage{multirow}
\usepackage{titlesec}
\usepackage{listings}
\usepackage{listingsutf8}
\usepackage{color}
\usepackage{enumitem}
\usepackage{cmap}
\usepackage{url}

\definecolor{green}{rgb}{0,0.6,0}
\definecolor{gray}{rgb}{0.5,0.5,0.5}
\definecolor{purple}{rgb}{0.58,0,0.82}

\lstdefinelanguage{none}{}

\lstset{
	language={Python},
	inputpath={../},
	backgroundcolor=\color{white},
	commentstyle=\color{green},
	keywordstyle=\color{blue},
	numberstyle=\color{gray}\scriptsize\ttfamily,
	stringstyle=\color{purple},
	basicstyle=\lst@ifdisplaystyle\footnotesize\fi\ttfamily,
	breakatwhitespace=false,
	breaklines=true,
	captionpos=b,
	keepspaces=true,
	numbers=left,
	numbersep=5pt,
	showspaces=false,
	showstringspaces=false,
	showtabs=false,
	tabsize=4,
	frame=single,
	morekeywords={},
	deletekeywords={},
	extendedchars=true,
	columns=fullflexible,
	inputencoding=utf8/cp1251,
	literate=%
		{~}{{\raise.25ex\hbox{$\mathtt{\sim}$}}}{1}
}

\titleformat*{\section}{\large\bfseries} 
\titleformat*{\subsection}{\normalsize\bfseries} 
\titleformat*{\subsubsection}{\normalsize\bfseries} 
\titleformat*{\paragraph}{\normalsize\bfseries} 
\titleformat*{\subparagraph}{\normalsize\bfseries} 

\counterwithin{figure}{section}
\counterwithin{equation}{section}
\counterwithin{table}{section}
\newcommand{\sign}[1][5cm]{\makebox[#1]{\hrulefill}}
\newcommand{\equipollence}{\quad\Leftrightarrow\quad}
\newcommand{\no}[1]{\overline{#1}}
\newcommand{\code}[1]{\lstinline[language=none]|#1|}
\newcommand{\data}[2]{
\paragraph{Описание.} #2.
\begin{figure}[H]
	\centering
	\includegraphics[width=\linewidth]{#1}
	\caption{Результат работы на данных #1.}
\end{figure}
}

\graphicspath{{../pics/}}
\captionsetup{justification=centering,margin=1cm}
\def\arraystretch{1.3}
\setlength\parindent{5ex}
\titlelabel{\thetitle.\quad}

\setitemize{topsep=0em, itemsep=0em}
\setenumerate{topsep=0em, itemsep=0em}


\begin{document}

\begin{titlepage}
\begin{center}
	Санкт-Петербургский Политехнический Университет Петра Великого\\[0.3cm]
	Институт компьютерных наук и технологий \\[0.3cm]
	Кафедра компьютерных систем и программных технологий\\[4cm]
	
	\textbf{ОТЧЕТ}\\ 
	\textbf{по лабораторной работе}\\[0.5cm]
	\textbf{<<Многокритериальная оптимизация>>}\\[0.1cm]
	Методы оптимизации и принятия решений\\[3.0cm]
\end{center}

\begin{flushright}
	\begin{minipage}{0.5\textwidth}
		\textbf{Работу выполнил студент}\\[3mm]
		гр. 3540901/91502 \hfill \sign[1.1cm] \hfill Дьячков В.В.\\[5mm]
		\textbf{Работу принял преподаватель}\\[5mm]
		\sign[3cm] \hfill к.т.н., доц. Сиднев А.Г. \\[5mm]
	\end{minipage}
\end{flushright}

\vfill

\begin{center}
	Санкт-Петербург\\[0.3cm]
	\the\year
\end{center}
\end{titlepage}

\addtocounter{page}{1}

\tableofcontents
\newpage

\section{Программа работы}

\begin{enumerate}
	\item Осуществить переход от многокритериальной задачи к однокритериальной с использованием следующих подходов.  Сформировать множество Парето из всех решений, полученных в результате выполнения этого пункта.
		\begin{enumerate}
			\item Выделение главного критерия
			\item Свертка критериев (аддитивная и мультипликативная)
			\item Максимин или минимакс (он же метод максиминной свертки)
			\item Метод последовательных уступок
			\item fgoalattain 
			\item Ведение метрики в пространстве критериев
		\end{enumerate}
	\item Выбрать одну из однокритериальных задач.
		\begin{enumerate}
			\item Привести подробную математическую модель выбранной задачи.
			\item Проанализировать полученное решение:\\
			Найти левые части ограничений задачи для полученного решения (численные значения).\\
			Обозначить активные ограничения задачи.\\
			Выбрать одно из активных ограничений задачи для преобразования в вероятностное: 
			$P\left( \sum_{j=1}^n a_{ij} x_j - b_i \leq 0 \right) \geq \alpha_i$.\\
			Решить задачу полученную задачу стохастического программирования, меняя $\alpha_i$ в следующем диапазоне: $0.1 \leq \alpha_i \leq 0.9$.\\
			Считать случайной величиной $b_i$ или элементы $\{a_{ij}\}$ $i$-й строки матрицы    (по выбору).
		\end{enumerate}
\end{enumerate}

\newpage

\section{Выполнение работы}


\section{Выводы}

В рамках данной работы:

\begin{itemize}
	\item 1
\end{itemize}

\end{document}
