\documentclass[a4paper,14pt]{extarticle}

\usepackage[utf8x]{inputenc}
\usepackage[T1,T2A]{fontenc}
\usepackage[russian]{babel}
\usepackage{hyperref}
\usepackage{indentfirst}
\usepackage{here}
\usepackage{array}
\usepackage{graphicx}
\usepackage{caption}
\usepackage{subcaption}
\usepackage{chngcntr}
\usepackage{amsmath}
\usepackage{amssymb}
\usepackage{amsthm}
\usepackage{pgfplots}
\usepackage{pgfplotstable}
\usepackage[left=2cm,right=2cm,top=2cm,bottom=2cm,bindingoffset=0cm]{geometry}
\usepackage{multicol}
\usepackage{askmaps}
\usepackage{titlesec}
\usepackage{listings}
\usepackage{color}
\usepackage{enumerate}
\usepackage{hhline}
\usepackage{enumitem}
\usepackage{courier}
\usepackage{wrapfig}
\usetikzlibrary{arrows,automata}

\setitemize{itemsep=0em}
\setenumerate{itemsep=0em}

\theoremstyle{definition}

\pgfkeys{/pgf/number format/.cd,1000 sep={\,}}

\definecolor{green}{rgb}{0,0.6,0}
\definecolor{gray}{rgb}{0.5,0.5,0.5}
\definecolor{purple}{rgb}{0.58,0,0.82}

\lstset{
	language=python,
	backgroundcolor=\color{white},   
	commentstyle=\color{green},
	keywordstyle=\color{blue},
	numberstyle=\tiny\color{gray},
	stringstyle=\color{purple},
	basicstyle=\footnotesize\ttfamily,
	breakatwhitespace=false,
	breaklines=true,
	captionpos=b,
	keepspaces=true,
	numbers=left,
	numbersep=5pt,
	showspaces=false,
	showstringspaces=false,
	showtabs=false,
	tabsize=2,
	frame=single,
	inputpath={../code/}
}

\renewcommand{\le}{\ensuremath{\leqslant}}
\renewcommand{\leq}{\ensuremath{\leqslant}}
\renewcommand{\ge}{\ensuremath{\geqslant}}
\renewcommand{\geq}{\ensuremath{\geqslant}}
\renewcommand{\epsilon}{\ensuremath{\varepsilon}}
\renewcommand{\phi}{\ensuremath{\varphi}}
\renewcommand{\thefigure}{\arabic{figure}} 	
\newcommand{\norm}[1]{\left\lVert#1\right\rVert}
\newcommand*\sfrac[2]{{}^{#1}\!/_{#2}}

%\titleformat*{\section}{\large\bfseries} 
\titleformat*{\subsection}{\normalsize\bfseries} 
\titleformat*{\subsubsection}{\normalsize\bfseries} 
\titleformat*{\paragraph}{\normalsize\bfseries} 
\titleformat*{\subparagraph}{\normalsize\bfseries} 

\counterwithin{figure}{section}
\counterwithin{equation}{section}
\counterwithin{table}{section}
\newcommand{\sign}[1][5cm]{\makebox[#1]{\hrulefill}}
\graphicspath{{../pics/}}
\captionsetup{justification=centering,margin=1cm}
\setlength\parindent{5ex}
\def\arraystretch{1.3}
\def\tabcolsep{12pt}
%\titlelabel{\thetitle.\quad}

\DeclareMathOperator*{\argmin}{argmin}
\DeclareMathOperator*{\argmax}{argmax}

\begin{document}

\begin{titlepage}
\begin{center}
	\textbf{Санкт-Петербургский Политехнический Университет \\Петра Великого}\\[0.3cm]
	\small Институт компьютерных наук и технологий \\[0.3cm]
	\small Кафедра компьютерных систем и программных технологий\\[4cm]
	
	\textbf{ОТЧЕТ}\\ \textbf{по расчетному заданию}\\[0.5cm]
	\textbf{<<Построение моделей>>}\\[0.1cm]
	\textbf{Системный анализ и принятие решений}\\[8.0cm]
\end{center}

\begin{flushright}
	\begin{minipage}{0.4\textwidth}
		\begin{flushleft}
			\small \textbf{Работу выполнил студент}\\[3mm]
			\small группа 33501/4 \hspace*{6mm} Дьячков В.В.\\[5mm]
			
			\small \textbf{Преподаватель}\\[5mm]
		 	\small \sign[3cm] \hspace*{5mm} Сабонис С.С.\\[0.5cm]
		\end{flushleft}
	\end{minipage}
\end{flushright}

\vfill

\begin{center}
	\small Санкт-Петербург\\
	\small \the\year
\end{center}
\end{titlepage}

\addtocounter{page}{1}

\tableofcontents
\listoffigures
%\listoftables
\newpage

\section{Техническое задание}

Решить задачу коммивояжера методом ветвей и границ в соответствии с матрицей по вариантам.  

Задача коммивояжера: имеется $n$ городов, задана матрица расстояний между городами. Коммивояжер должен побывать в каждом городе только один раз и вернуться в начальный город. Требуется найти маршрут, имеющий минимальную длину.

\section{Исходные данные}

\begin{table}[H]
\begin{center}
	\def\tabcolsep{15pt}
	\caption{Таблица переходов}
	\label{tab:1}
	\begin{tabular}{|c||c|c|c|c|c|c|c|c|}
		\hline
		Из \textbackslash В & $1$ & $2$ & $3$ & $4$ & $5$ & $6$ \\
		\hhline{|=#=|=|=|=|=|=|}
		$1$ & $\infty$ & $14$ & $19$ & $10$ & $19$ & $17$ \\
		\hline
		$2$ & $11$ & $\infty$ & $33$ & $8$ & $25$ & $31$ \\
		\hline
		$3$ & $16$ & $24$ & $\infty$ & $28$ & $8$ & $10$ \\ 
		\hline
		$4$ & $8$ & $7$ & $23$ & $\infty$ & $21$ & $26$ \\
		\hline
		$5$ & $15$ & $20$ & $7$ & $21$ & $\infty$ & $16$ \\
		\hline
		$6$ & $12$ & $22$ & $7$ & $19$ & $12$ & $\infty$ \\ 
		\hline
	\end{tabular}
\end{center}
\end{table} 

\section{Решение методом ветвей и границ}

\textbf{1.} Найдем минимум в каждой строки исходной таблицы \ref{tab:1}.

\begin{table}[H]
\begin{center}
	\def\tabcolsep{15pt}
	\caption{Таблица переходов}
	\label{tab:2}
	\begin{tabular}{|c||c|c|c|c|c|c|c|c|}
		\hline
		Из \textbackslash В & $1$ & $2$ & $3$ & $4$ & $5$ & $6$ & $\min$ \\
		\hhline{|=#=|=|=|=|=|=#=|}
		$1$ & $\infty$ & $14$ & $19$ & $10$ & $19$ & $17$ & $10$ \\
		\hline
		$2$ & $11$ & $\infty$ & $33$ & $8$ & $25$ & $31$ & $8$ \\
		\hline
		$3$ & $16$ & $24$ & $\infty$ & $28$ & $8$ & $10$ & $8$ \\ 
		\hline
		$4$ & $8$ & $7$ & $23$ & $\infty$ & $21$ & $26$ & $7$ \\
		\hline
		$5$ & $15$ & $20$ & $7$ & $21$ & $\infty$ & $16$ & $7$ \\
		\hline
		$6$ & $12$ & $22$ & $7$ & $19$ & $12$ & $\infty$ & $7$ \\ 
		\hline
	\end{tabular}
\end{center}
\end{table} 

Вычтем найденные минимумы из каждой строки.

\begin{table}[H]
\begin{center}
	\def\tabcolsep{15pt}
	\caption{Таблица переходов}
	\label{tab:3}
	\begin{tabular}{|c||c|c|c|c|c|c|c|c|}
		\hline
		Из \textbackslash В & $1$ & $2$ & $3$ & $4$ & $5$ & $6$ \\
		\hhline{|=#=|=|=|=|=|=|}
		$1$ & $\infty$ & $4$ & $9$ & $0$ & $9$ & $7$ \\
		\hline
		$2$ & $3$ & $\infty$ & $25$ & $0$ & $17$ & $23$ \\
		\hline
		$3$ & $8$ & $16$ & $\infty$ & $20$ & $0$ & $2$ \\ 
		\hline
		$4$ & $1$ & $0$ & $16$ & $\infty$ & $14$ & $19$ \\
		\hline
		$5$ & $8$ & $13$ & $0$ & $14$ & $\infty$ & $9$ \\
		\hline
		$6$ & $5$ & $15$ & $0$ & $12$ & $5$ & $\infty$ \\ 
		\hhline{|=#=|=|=|=|=|=|}
		$\min$ & $1$ & $0$ & $0$ & $0$ & $0$ & $2$ \\ 
		\hline
	\end{tabular}
\end{center}
\end{table}

Из тех столбцов, в которых не оказалось нулей, вычтем минимальный элемент.

\begin{table}[H]
\begin{center}
	\def\tabcolsep{15pt}
	\caption{Таблица переходов}
	\label{tab:4}
	\begin{tabular}{|c||c|c|c|c|c|c|c|c|}
		\hline
		Из \textbackslash В & $1$ & $2$ & $3$ & $4$ & $5$ & $6$ \\
		\hhline{|=#=|=|=|=|=|=|}
		$1$ & $\infty$ & $4$ & $9$ & $0$ & $9$ & $5$ \\
		\hline
		$2$ & $2$ & $\infty$ & $25$ & $0$ & $17$ & $21$ \\
		\hline
		$3$ & $7$ & $16$ & $\infty$ & $20$ & $0$ & $0$ \\ 
		\hline
		$4$ & $0$ & $0$ & $16$ & $\infty$ & $14$ & $17$ \\
		\hline
		$5$ & $7$ & $13$ & $0$ & $14$ & $\infty$ & $7$ \\
		\hline
		$6$ & $4$ & $15$ & $0$ & $12$ & $5$ & $\infty$ \\ 
		\hline
	\end{tabular}
\end{center}
\end{table}

Сумма элементов, которые мы вычитали, равняется $50$. Следовательно оптимальный путь не может быть меньше $50$:
\begin{equation*}
	h = 50,\ V(H) = h = 50
\end{equation*}

\textbf{2.} Заменим в таблице \ref{tab:4} нули на сумму минимального элемента строки и минимального элемента столбца.

\begin{table}[H]
\begin{center}
	\def\tabcolsep{15pt}
	\caption{Таблица переходов}
	\label{tab:5}
	\begin{tabular}{|c||c|c|c|c|c|c|c|c|}
		\hline
		Из \textbackslash В & $1$ & $2$ & $3$ & $4$ & $5$ & $6$ \\
		\hhline{|=#=|=|=|=|=|=|}
		$1$ & $-$ & $-$ & $-$ & $4$ & $-$ & $-$ \\
		\hline
		$2$ & $-$ & $-$ & $-$ & $2$ & $-$ & $-$ \\
		\hline
		$3$ & $-$ & $-$ & $-$ & $-$ & $5$ & $5$ \\ 
		\hline
		$4$ & $2$ & $4$ & $-$ & $-$ & $-$ & $-$ \\
		\hline
		$5$ & $-$ & $-$ & \textcolor{red}{\boldmath$7$} & $-$ & $-$ & $-$ \\
		\hline
		$6$ & $-$ & $-$ & $4$ & $-$ & $-$ & $-$ \\ 
		\hline
	\end{tabular}
\end{center}
\end{table}

\begin{gather*}
\max = 7\ (5 \rightarrow 3) \\
G^{(0)} = G_{5,3}^{(0)} \cup G_{\overline{5,3}}^{(0)} \\
V(G_{\overline{5,3}}^{(0)}) = V(H) + 7 = 50 + 7 = 57
\end{gather*}

Вычеркнем из таблицы  \ref{tab:4} строку $5$ и столбец $3$, при этом запретим путь $3 \rightarrow 5$. Найдем минимумы для каждого столбца и каждой строки.

\begin{table}[H]
\begin{center}
	\def\tabcolsep{15pt}
	\caption{Таблица переходов}
	\label{tab:6}
	\begin{tabular}{|c||c|c|c|c|c|c|c|}
		\hline
		Из \textbackslash В & $1$ & $2$ & $4$ & $5$ & $6$ & $\min$ \\
		\hhline{|=#=|=|=|=|=|=|}
		$1$ & $\infty$ & $4$ & $0$ & $9$ & $5$ & $0$ \\
		\hline
		$2$ & $2$ & $\infty$ & $0$ & $17$ & $21$ & $0$ \\
		\hline
		$3$ & $7$ & $16$ & $20$ & $\infty$ & $0$ & $0$ \\ 
		\hline
		$4$ & $0$ & $0$ & $\infty$ & $14$ & $17$ & $0$ \\
		\hline
		$6$ & $4$ & $15$ & $12$ & $5$ & $\infty$ & $4$ \\
		\hline
	\end{tabular}
\end{center}
\end{table}

Вычтем из строки $6$ минимальный элемент, равный $4$, т.к. она не содержит ноль. В полученной таблице столбец $5$ так же не содержит ноль, поэтому вычтем и из него наименьший элемент, равный $5-4 = 1$.

\begin{table}[H]
\begin{center}
	\def\tabcolsep{15pt}
	\caption{Таблица переходов}
	\label{tab:7}
	\begin{tabular}{|c||c|c|c|c|c|c|c|}
		\hline
		Из \textbackslash В & $1$ & $2$ & $4$ & $5$ & $6$ & $\min$ \\
		\hhline{|=#=|=|=|=|=|=|}
		$1$ & $\infty$ & $4$ & $0$ & $8$ & $5$ & $0$ \\
		\hline
		$2$ & $2$ & $\infty$ & $0$ & $16$ & $21$ & $0$ \\
		\hline
		$3$ & $7$ & $16$ & $20$ & $\infty$ & $0$ & $0$ \\ 
		\hline
		$4$ & $0$ & $0$ & $\infty$ & $13$ & $17$ & $0$ \\
		\hline
		$6$ & $0$ & $11$ & $8$ & $0$ & $\infty$ & $4$ \\
		\hhline{|=#=|=|=|=|=|=|} 
		$\min$ & $0$ & $0$ & $0$ & $1$ & $0$ & \\ 
		\hline
	\end{tabular}
\end{center}
\end{table}

Сумма вычитаемых элементов равна $5$, следовательно:
\begin{equation*}
V(G_{5,3}^{(0)}) = V(H) + 5 = 50 + 5 = 55 < 57 = V(G_{\overline{5,3}}^{(0)}) \Rightarrow V(G^{(0)}) = 55
\end{equation*}

\textbf{3.1} Допустим нет перехода $5 \rightarrow 3$. Тогда заменим его на бесконечность, а из строки и столбца, содержащих данный переход, вычтем минимальные элементы: из строки вычтем $7$, а из столбца $0$.

\begin{table}[H]
\begin{center}
	\def\tabcolsep{15pt}
	\caption{Таблица переходов}
	\label{tab:8}
	\begin{tabular}{|c||c|c|c|c|c|c|c|c|}
		\hline
		Из \textbackslash В & $1$ & $2$ & $3$ & $4$ & $5$ & $6$ \\
		\hhline{|=#=|=|=|=|=|=|}
		$1$ & $\infty$ & $4$ & $9$ & $0$ & $9$ & $5$ \\
		\hline
		$2$ & $2$ & $\infty$ & $25$ & $0$ & $17$ & $21$ \\
		\hline
		$3$ & $7$ & $16$ & $\infty$ & $20$ & $0$ & $0$ \\ 
		\hline
		$4$ & $0$ & $0$ & $16$ & $\infty$ & $14$ & $17$ \\
		\hline
		$5$ & $0$ & $6$ & $\infty$ & $7$ & $\infty$ & $0$ \\
		\hline
		$6$ & $4$ & $15$ & $0$ & $12$ & $5$ & $\infty$ \\ 
		\hline
	\end{tabular}
\end{center}
\end{table}

Заменим в таблице \ref{tab:8} нули на сумму минимального элемента строки и минимального элемента столбца.

\begin{table}[H]
\begin{center}
	\def\tabcolsep{15pt}
	\caption{Таблица переходов}
	\label{tab:9}
	\begin{tabular}{|c||c|c|c|c|c|c|c|c|}
		\hline
		Из \textbackslash В & $1$ & $2$ & $3$ & $4$ & $5$ & $6$ \\
		\hhline{|=#=|=|=|=|=|=|}
		$1$ & $-$ & $-$ & $-$ & $4$ & $-$ & $-$ \\
		\hline
		$2$ & $-$ & $-$ & $-$ & $2$ & $-$ & $-$ \\
		\hline
		$3$ & $-$ & $-$ & $-$ & $-$ & $5$ & $0$ \\ 
		\hline
		$4$ & $0$ & $4$ & $-$ & $-$ & $-$ & $-$ \\
		\hline
		$5$ & $0$ & $-$ & $-$ & $-$ & $-$ & $0$ \\
		\hline
		$6$ & $-$ & $-$ & \textcolor{red}{\boldmath$13$} & $-$ & $-$ & $-$ \\ 
		\hline
	\end{tabular}
\end{center}
\end{table}

\begin{gather*}
\max = 13\ (6 \rightarrow 3) \\
G^{(1)} = G_{6,3}^{(1)} \cup G_{\overline{6,3}}^{(1)} \\
V(G_{\overline{6,3}}^{(1)}) = V(G^{(0)}) + 13 = 55 + 13 = 68
\end{gather*}

Вычеркнем из таблицы $6$ строку и $3$ столбец, при этом запретим путь $3 \rightarrow 6$. Найдем минимумы для каждого столбца и каждой строки.

\begin{table}[H]
\begin{center}
	\def\tabcolsep{15pt}
	\caption{Таблица переходов}
	\label{tab:10}
	\begin{tabular}{|c||c|c|c|c|c|c|c|c|}
		\hline
		Из \textbackslash В & $2$ & $3$ & $4$ & $5$ & $6$ & $\min$ \\
		\hhline{|=#=|=|=|=|=|=|}
		$1$ & $4$ & $9$ & $0$ & $9$ & $5$ & $0$ \\
		\hline
		$2$ & $\infty$ & $25$ & $0$ & $17$ & $21$ & $0$ \\
		\hline
		$3$ & $16$ & $\infty$ & $20$ & $0$ & $\infty$ & $0$ \\ 
		\hline
		$4$ & $0$ & $16$ & $\infty$ & $14$ & $17$ & $0$ \\
		\hline
		$5$ & $6$ & $\infty$ & $7$ & $\infty$ & $0$ & $0$ \\
		\hhline{|=#=|=|=|=|=|=|}
		$\min$ & $0$ & $9$ & $0$ & $8$ & $0$ & \\ 
		\hline
	\end{tabular}
\end{center}
\end{table}

Сумма вычитаемых элементов равна $9$, следовательно:
\begin{equation*}
V(G_{6,3}^{(1)}) = V(G^{(0)}) + 9 = 64 < 68 = V(G_{\overline{3,6}}^{(1)}) \Rightarrow V(G^{(0)}) = 64
\end{equation*}

\textbf{4.1} Заменим в таблице \ref{tab:10} нули на сумму минимального элемента строки и минимального элемента столбца. 

\begin{table}[H]
\begin{center}
	\def\tabcolsep{15pt}
	\caption{Таблица переходов}
	\label{tab:11}
	\begin{tabular}{|c||c|c|c|c|c|c|c|c|}
		\hline
		Из \textbackslash В & $2$ & $3$ & $4$ & $5$ & $6$ \\
		\hhline{|=#=|=|=|=|=|=|}
		$1$ & $-$ & $-$ & $4$ & $-$ & $-$ \\
		\hline
		$2$ & $-$ & $-$ & $17$ & $-$ & $-$ \\
		\hline
		$3$ & $-$ & $-$ & $-$ & \redbold{$25$} & $-$ \\ 
		\hline
		$4$ & $18$ & $-$ & $-$ & $-$ & $-$ \\
		\hline
		$5$ & $-$ & $-$ & $-$ & $-$ & $11$ \\
		\hline
	\end{tabular}
\end{center}
\end{table}

\begin{gather*}
\max = 25\ (3 \rightarrow 5) \\
G^{(1)} = G_{3,5}^{(1)} \cup G_{\overline{3,5}}^{(1)} \\
V(G_{\overline{3,5}}^{(1)}) = V(G^{(0)}) + 25 = 64 + 25 = 89
\end{gather*}

Вычеркнем из таблицы $3$ строку и $5$ столбец, при этом запретим путь $5 \rightarrow 6$. Найдем минимумы для каждого столбца и каждой строки.

\begin{table}[H]
\begin{center}
	\def\tabcolsep{15pt}
	\caption{Таблица переходов}
	\label{tab:10}
	\begin{tabular}{|c||c|c|c|c|c|c|c|c|}
		\hline
		Из \textbackslash В & $2$ & $3$ & $4$ & $6$ & $\min$ \\
		\hhline{|=#=|=|=|=|=|=|}
		$1$ & $4$ & $9$ & $0$ & $5$ & $0$ \\
		\hline
		$2$ & $\infty$ & $25$ & $0$ & $21$ & $0$ \\
		\hline
		$4$ & $0$ & $16$ & $\infty$ & $17$ & $0$ \\
		\hline
		$5$ & $6$ & $\infty$ & $7$ & $\infty$ & $6$ \\
		\hline
	\end{tabular}
\end{center}
\end{table}

Вычтем из строки $5$ минимальный элемент, равный $6$, после чего вычтем минимальный элемент из столбца $3$, равный $9$.

\begin{table}[H]
\begin{center}
	\def\tabcolsep{15pt}
	\caption{Таблица переходов}
	\label{tab:10}
	\begin{tabular}{|c||c|c|c|c|c|c|c|c|}
		\hline
		Из \textbackslash В & $2$ & $3$ & $4$ & $6$ & $\min$ \\
		\hhline{|=#=|=|=|=|=|=|}
		$1$ & $4$ & $0$ & $0$ & $5$ & $0$ \\
		\hline
		$2$ & $\infty$ & $16$ & $0$ & $21$ & $0$ \\
		\hline
		$4$ & $0$ & $7$ & $\infty$ & $17$ & $0$ \\
		\hline
		$5$ & $0$ & $\infty$ & $1$ & $\infty$ & $6$ \\
		\hhline{|=#=|=|=|=|=|=|}
		$\min$ & $0$ & $9$ & $0$ & $0$ & \\ 
		\hline
	\end{tabular}
\end{center}
\end{table}

Сумма вычитаемых элементов равна $15$, следовательно:
\begin{equation*}
V(G_{3,5}^{(1)}) = V(G^{(0)}) + 15 = 79 < 89 = V(G_{\overline{3,6}}^{(1)}) \Rightarrow V(G^{(1)}) = 79
\end{equation*}

%%%%%%%%%%%%%%%%%%%%%%%%%%%%%%%%%%%%%%%%%

\textbf{4.2} Допустим нет перехода $6 \rightarrow 3$. Тогда в таблице \ref{tab:8} заменим его на бесконечность, а из строки и столбца, содержащих данный переход, вычтем минимальные элементы: из строки вычтем $4$, а из столбца $9$.

\begin{table}[H]
\begin{center}
	\def\tabcolsep{15pt}
	\caption{Таблица переходов}
	\label{tab:11}
	\begin{tabular}{|c||c|c|c|c|c|c|c|c|}
		\hline
		Из \textbackslash В & $1$ & $2$ & $3$ & $4$ & $5$ & $6$ \\
		\hhline{|=#=|=|=|=|=|=|}
		$1$ & $\infty$ & $4$ & $0$ & $0$ & $9$ & $5$ \\
		\hline
		$2$ & $2$ & $\infty$ & $16$ & $0$ & $17$ & $21$ \\
		\hline
		$3$ & $7$ & $16$ & $\infty$ & $20$ & $0$ & $0$ \\ 
		\hline
		$4$ & $0$ & $0$ & $7$ & $\infty$ & $14$ & $17$ \\
		\hline
		$5$ & $0$ & $6$ & $\infty$ & $7$ & $\infty$ & $0$ \\
		\hline
		$6$ & $0$ & $11$ & $\infty$ & $8$ & $1$ & $\infty$ \\ 
		\hline
	\end{tabular}
\end{center}
\end{table}

Заменим в таблице \ref{tab:11} нули на сумму минимального элемента строки и минимального элемента столбца.

\begin{table}[H]
\begin{center}
	\def\tabcolsep{15pt}
	\caption{Таблица переходов}
	\label{tab:11}
	\begin{tabular}{|c||c|c|c|c|c|c|c|c|}
		\hline
		Из \textbackslash В & $1$ & $2$ & $3$ & $4$ & $5$ & $6$ \\
		\hhline{|=#=|=|=|=|=|=|}
		$1$ & $-$ & $-$ & \redbold{$7$} & $0$ & $-$ & $-$ \\
		\hline
		$2$ & $-$ & $-$ & $-$ & $2$ & $-$ & $-$ \\
		\hline
		$3$ & $-$ & $-$ & $-$ & $-$ & $1$ & $0$ \\ 
		\hline
		$4$ & $0$ & $4$ & $-$ & $-$ & $-$ & $-$ \\
		\hline
		$5$ & $0$ & $-$ & $-$ & $-$ & $-$ & $0$ \\
		\hline
		$6$ & $1$ & $-$ & $-$ & $-$ & $-$ & $-$ \\ 
		\hline
	\end{tabular}
\end{center}
\end{table}

\begin{gather*}
\max = 7\ (1 \rightarrow 3) \\
G^{(2)} = G_{1,3}^{(2)} \cup G_{\overline{1,3}}^{(2)} \\
V(G_{\overline{1,3}}^{(2)}) = V(G^{(1)}) + 7 = 68 + 7 = 75
\end{gather*}

Вычеркнем из таблицы $1$ строку и $3$ столбец, при этом запретим путь $3 \rightarrow 1$. Найдем минимумы для каждого столбца и каждой строки.

\begin{table}[H]
\begin{center}
	\def\tabcolsep{15pt}
	\caption{Таблица переходов}
	\label{tab:11}
	\begin{tabular}{|c||c|c|c|c|c|c|c|c|}
		\hline
		Из \textbackslash В & $1$ & $2$ & $4$ & $5$ & $6$ & $\min$ \\
		\hhline{|=#=|=|=|=|=|=|}
		\hline
		$2$ & $2$ & $\infty$ & $0$ & $17$ & $21$ & $0$ \\
		\hline
		$3$ & $\infty$ & $16$ & $20$ & $0$ & $0$ & $0$  \\ 
		\hline
		$4$ & $0$ & $0$ & $\infty$ & $14$ & $17$ & $0$  \\
		\hline
		$5$ & $0$ & $6$ & $7$ & $\infty$ & $0$ & $0$  \\
		\hline
		$6$ & $0$ & $11$ & $8$ & $1$ & $\infty$ & $0$  \\ 
		\hhline{|=#=|=|=|=|=|=|}
		$\min$ & $0$ & $0$ & $0$ & $0$ & $0$ & \\
		\hline 
	\end{tabular}
\end{center}
\end{table}

Сумма вычитаемых элементов равна $0$, следовательно:
\begin{equation*}
V(G_{1,3}^{(2)}) = V(G^{(1)}) + 0 = 68 < 75 = V(G_{\overline{1,3}}^{(2)}) \Rightarrow V(G^{(2)}) = 68
\end{equation*}

Заменим в таблице \ref{tab:11} нули на сумму минимального элемента строки и минимального элемента столбца.

\begin{table}[H]
\begin{center}
	\def\tabcolsep{15pt}
	\caption{Таблица переходов}
	\label{tab:11}
	\begin{tabular}{|c||c|c|c|c|c|c|c|c|}
		\hline
		Из \textbackslash В & $1$ & $2$ & $4$ & $5$ & $6$ \\
		\hhline{|=#=|=|=|=|=|=|}
		\hline
		$2$ & $-$ & $-$ & \redbold{$9$} & $-$ & $-$ \\
		\hline
		$3$ & $-$ & $-$ & $-$ & $1$ & $0$ \\ 
		\hline
		$4$ & $0$ & $6$ & $-$ & $-$ & $-$ \\
		\hline
		$5$ & $0$ & $-$ & $-$ & $-$ & $0$ \\
		\hline
		$6$ & $1$ & $-$ & $-$ & $-$ & $-$ \\
		\hline 
	\end{tabular}
\end{center}
\end{table}

\begin{gather*}
\max = 9\ (2 \rightarrow 4) \\
G^{(3)} = G_{2,4}^{(3)} \cup G_{\overline{2,4}}^{(3)} \\
V(G_{\overline{2,4}}^{(3)}) = V(G^{(2)}) + 9 = 68 + 9 = 77
\end{gather*}

Вычеркнем из таблицы $2$ строку и $4$ столбец, при этом запретим путь $4 \rightarrow 2$. Найдем минимумы для каждого столбца и каждой строки.

\begin{table}[H]
\begin{center}
	\def\tabcolsep{15pt}
	\caption{Таблица переходов}
	\label{tab:11}
	\begin{tabular}{|c||c|c|c|c|c|c|c|c|}
		\hline
		Из \textbackslash В & $1$ & $2$ & $5$ & $6$ & $\min$ \\
		\hhline{|=#=|=|=|=|=|=|}
		\hline
		$3$ & $\infty$ & $16$ & $0$ & $0$ & $0$  \\ 
		\hline
		$4$ & $0$ & $\infty$ & $14$ & $17$ & $0$  \\
		\hline
		$5$ & $0$ & $6$ & $\infty$ & $0$ & $0$  \\
		\hline
		$6$ & $0$ & $11$ & $1$ & $\infty$ & $0$  \\ 
		\hhline{|=#=|=|=|=|=|=|}
		$\min$ & $0$ & $6$ & $0$ & $0$ & \\
		\hline 
	\end{tabular}
\end{center}
\end{table}

Вычтем из столбца $2$ минимальный элемент, равный $6$.

\begin{table}[H]
\begin{center}
	\def\tabcolsep{15pt}
	\caption{Таблица переходов}
	\label{tab:11}
	\begin{tabular}{|c||c|c|c|c|c|c|c|c|}
		\hline
		Из \textbackslash В & $1$ & $2$ & $5$ & $6$ & $\min$ \\
		\hhline{|=#=|=|=|=|=|=|}
		\hline
		$3$ & $\infty$ & $10$ & $0$ & $0$ & $0$  \\ 
		\hline
		$4$ & $0$ & $\infty$ & $14$ & $17$ & $0$  \\
		\hline
		$5$ & $0$ & $0$ & $\infty$ & $0$ & $0$  \\
		\hline
		$6$ & $0$ & $5$ & $1$ & $\infty$ & $0$  \\ 
		\hhline{|=#=|=|=|=|=|=|}
		$\min$ & $0$ & $6$ & $0$ & $0$ & \\
		\hline 
	\end{tabular}
\end{center}
\end{table}

Сумма вычитаемых элементов равна $6$, следовательно:
\begin{equation*}
V(G_{2,4}^{(3)}) = V(G^{(2)}) + 6 = 74 < 77 = V(G_{\overline{2,4}}^{(3)}) \Rightarrow V(G^{(3)}) = 74
\end{equation*}

%%%%%%%%%%%%%%%%%%%%%%%%%%%%%%%%%%%%%%%%

\textbf{3.2} Заменим в таблице \ref{tab:11} нули на сумму минимального элемента строки и минимального элемента столбца.

\begin{table}[H]
\begin{center}
	\def\tabcolsep{15pt}
	\caption{Таблица переходов}
	\label{tab:8}
	\begin{tabular}{|c||c|c|c|c|c|c|}
		\hline
		Из \textbackslash В & $1$ & $2$ & $4$ & $5$ & $6$ \\
		\hhline{|=#=|=|=|=|=|}
		$1$ & $-$ & $-$ & $4$ & $-$ & $-$ \\
		\hline
		$2$ & $-$ & $-$ & $2$ & $-$ & $-$ \\
		\hline
		$3$ & $-$ & $-$ & $-$ & $-$ & \redbold{$12$} \\ 
		\hline
		$4$ & $0$ & $4$ & $-$ & $-$ & $-$ \\
		\hline
		$6$ & $0$ & $-$ & $-$ & $8$ & $-$ \\
		\hline
	\end{tabular}
\end{center}
\end{table}

\begin{gather*}
\max = 12\ (3 \rightarrow 6) \\
G^{(1)} = G_{3,6}^{(1)} \cup G_{\overline{3,6}}^{(1)} \\
V(G_{\overline{3,6}}^{(1)}) = V(G^{(0)}) + 12 = 55 + 12 = 67
\end{gather*}

Вычеркнем из таблицы $3$ строку и $6$ столбец, при этом запретим путь $6 \rightarrow 3$. Найдем минимумы для каждого столбца и каждой строки.

\begin{table}[H]
\begin{center}
	\def\tabcolsep{15pt}
	\caption{Таблица переходов}
	\label{tab:9}
	\begin{tabular}{|c||c|c|c|c|c|c|}
		\hline
		Из \textbackslash В & $1$ & $2$ & $4$ & $5$ & $\min$ \\
		\hhline{|=#=|=|=|=|=|}
		$1$ & $\infty$ & $4$ & $0$ & $8$ & $0$ \\
		\hline
		$2$ & $2$ & $\infty$ & $0$ & $16$ & $0$ \\
		\hline
		$4$ & $0$ & $0$ & $\infty$ & $13$ & $0$ \\
		\hline
		$6$ & $0$ & $11$ & $8$ & $\infty$ & $0$ \\
		\hhline{|=#=|=|=|=|=|}
		$\min$ & $0$ & $0$ & $0$ & $8$ & \\ 
		\hline
	\end{tabular}
\end{center}
\end{table}

Вычтем из столбца $5$ минимальный элемент, равный $8$.

\begin{table}[H]
\begin{center}
	\def\tabcolsep{15pt}
	\caption{Таблица переходов}
	\label{tab:10}
	\begin{tabular}{|c||c|c|c|c|c|c|}
		\hline
		Из \textbackslash В & $1$ & $2$ & $4$ & $5$ & $\min$ \\
		\hhline{|=#=|=|=|=|=|}
		$1$ & $\infty$ & $4$ & $0$ & $0$ & $0$ \\
		\hline
		$2$ & $2$ & $\infty$ & $0$ & $8$ & $0$ \\
		\hline
		$4$ & $0$ & $0$ & $\infty$ & $5$ & $0$ \\
		\hline
		$6$ & $0$ & $11$ & $8$ & $\infty$ & $0$ \\
		\hhline{|=#=|=|=|=|=|}
		$\min$ & $0$ & $0$ & $0$ & $8$ & \\ 
		\hline
	\end{tabular}
\end{center}
\end{table}

Сумма вычитаемых элементов равна $8$, следовательно:
\begin{equation*}
V(G_{3,6}^{(1)}) = V(G^{(0)}) + 8 = 63 < 67 = V(G_{\overline{3,6}}^{(1)}) \Rightarrow V(G^{(1)}) = 63
\end{equation*}

%%%%%%%%%%%%%%%%%%%%%%%%%%%%%

\textbf{4.1.} Допустим нет перехода $3 \rightarrow 6$. Тогда в таблице \ref{tab:7} заменим его на бесконечность, а из строки и столбца, содержащих данный переход, вычтем минимальные элементы: из строки вычтем $7$, а из столбца $5$.

\begin{table}[H]
\begin{center}
	\def\tabcolsep{15pt}
	\caption{Таблица переходов}
	\label{tab:111}
	\begin{tabular}{|c||c|c|c|c|c|c|c|}
		\hline
		Из \textbackslash В & $1$ & $2$ & $4$ & $5$ & $6$ \\
		\hhline{|=#=|=|=|=|=|=|}
		$1$ & $\infty$ & $4$ & $0$ & $8$ & $0$ \\
		\hline
		$2$ & $2$ & $\infty$ & $0$ & $16$ & $16$ \\
		\hline
		$3$ & $0$ & $9$ & $13$ & $\infty$ & $\infty$ \\ 
		\hline
		$4$ & $0$ & $0$ & $\infty$ & $13$ & $12$ \\
		\hline
		$6$ & $0$ & $11$ & $8$ & $0$ & $\infty$ \\
		\hline
	\end{tabular}
\end{center}
\end{table}

Заменим в таблице \ref{tab:111} нули на сумму минимального элемента строки и минимального элемента столбца. 

\begin{table}[H]
\begin{center}
	\def\tabcolsep{15pt}
	\caption{Таблица переходов}
	\label{tab:111}
	\begin{tabular}{|c||c|c|c|c|c|c|c|}
		\hline
		Из \textbackslash В & $1$ & $2$ & $4$ & $5$ & $6$ \\
		\hhline{|=#=|=|=|=|=|=|}
		$1$ & $-$ & $-$ & $0$ & $-$ & \redbold{$12$} \\
		\hline
		$2$ & $-$ & $-$ & $2$ & $-$ & $-$ \\
		\hline
		$3$ & $9$ & $-$ & $-$ & $-$ & $-$ \\ 
		\hline
		$4$ & $0$ & $4$ & $-$ & $-$ & $-$ \\
		\hline
		$6$ & $0$ & $-$ & $-$ & $8$ & $-$ \\
		\hline
	\end{tabular}
\end{center}
\end{table}

\begin{gather*}
\max = 12\ (1 \rightarrow 6) \\
G^{(2)} = G_{1,6}^{(2)} \cup G_{\overline{1,6}}^{(2)} \\
V(G_{\overline{1,6}}^{(2)}) = V(G^{(1)}) + 6 = 67 + 6 = 73
\end{gather*}

Вычеркнем из таблицы \ref{tab:111} строку $1$ и столбец $6$, при этом запретим переход $6 \rightarrow 1$. Найдем минимумы для каждого столбца и каждой строки.

\begin{table}[H]
\begin{center}
	\def\tabcolsep{15pt}
	\caption{Таблица переходов}
	\label{tab:112}
	\begin{tabular}{|c||c|c|c|c|c|c|c|}
		\hline
		Из \textbackslash В & $1$ & $2$ & $4$ & $5$ & $\min$ \\
		\hhline{|=#=|=|=|=|=|=|}
		$2$ & $2$ & $\infty$ & $0$ & $16$ & $0$ \\
		\hline
		$3$ & $0$ & $9$ & $13$ & $\infty$ & $0$  \\ 
		\hline
		$4$ & $0$ & $0$ & $\infty$ & $13$ & $0$  \\
		\hline
		$6$ & $\infty$ & $11$ & $8$ & $0$ & $0$  \\
		\hhline{|=#=|=|=|=|=|}
		$\min$ & $0$ & $0$ & $0$ & $0$ & \\ 
		\hline
	\end{tabular}
\end{center}
\end{table}

Сумма вычитаемых элементов равна $0$, следовательно:
\begin{equation*}
V(G_{1,6}^{(2)}) = V(G^{(1)}) + 0 = 67 < 73 = V(G_{\overline{1,6}}^{(2)})\Rightarrow V(G^{(2)}) = 67
\end{equation*}

\textbf{4.2ff.} Заменим в таблице \ref{tab:112} нули на сумму минимального элемента строки и минимального элемента столбца. 

\begin{table}[H]
\begin{center}
	\def\tabcolsep{15pt}
	\caption{Таблица переходов}
	\label{tab:112}
	\begin{tabular}{|c||c|c|c|c|c|c|c|}
		\hline
		Из \textbackslash В & $1$ & $2$ & $4$ & $5$ \\
		\hhline{|=#=|=|=|=|=|=|}
		$2$ & $-$ & $-$ & $10$ & $-$ \\
		\hline
		$3$ & $9$ & $-$ & $-$ & $-$ \\ 
		\hline
		$4$ & $0$ & $9$ & $-$ & $-$ \\
		\hline
		$6$ & $-$ & $-$ & $-$ & \redbold{$21$} \\
		\hline
	\end{tabular}
\end{center}
\end{table}

\begin{gather*}
\max = 21\ (6 \rightarrow 5) \\
G^{(3)} = G_{6,5}^{(3)} \cup G_{\overline{6,5}}^{(3)} \\
V(G_{\overline{6,5}}^{(3)}) = V(G^{(2)}) + 21 = 67 + 21 = 88
\end{gather*}

Вычеркнем из таблицы \ref{tab:112} строку $6$ и столбец $5$, при этом запретим переход $3 \rightarrow 1$. Найдем минимумы для каждого столбца и каждой строки.

\begin{table}[H]
\begin{center}
	\def\tabcolsep{15pt}
	\caption{Таблица переходов}
	\label{tab:112}
	\begin{tabular}{|c||c|c|c|c|c|c|c|}
		\hline
		Из \textbackslash В & $1$ & $2$ & $4$ & $\min$ \\
		\hhline{|=#=|=|=|=|}
		$2$ & $2$ & $\infty$ & $0$ & $0$ \\
		\hline
		$3$ & $\infty$ & $9$ & $13$ & $9$  \\ 
		\hline
		$4$ & $0$ & $0$ & $\infty$ & $0$  \\
		\hline
	\end{tabular}
\end{center}
\end{table}

Вычтем из строки $3$ минимальный элемент, равный $9$.

\begin{table}[H]
\begin{center}
	\def\tabcolsep{15pt}
	\caption{Таблица переходов}
	\label{tab:112}
	\begin{tabular}{|c||c|c|c|c|c|c|c|}
		\hline
		Из \textbackslash В & $1$ & $2$ & $4$ & $\min$ \\
		\hhline{|=#=|=|=|=|}
		$2$ & $2$ & $\infty$ & $0$ & $0$ \\
		\hline
		$3$ & $\infty$ & $0$ & $4$ & $0$  \\ 
		\hline
		$4$ & $0$ & $0$ & $\infty$ & $0$  \\
		\hhline{|=#=|=|=|=|}
		$\min$ & $0$ & $0$ & $0$ & \\ 
		\hline
	\end{tabular}
\end{center}
\end{table}

Сумма вычитаемых элементов равна $9$, следовательно:
\begin{equation*}
V(G_{6,5}^{(3)}) = V(G^{(2)}) + 9 = 76 < 88 = V(G_{\overline{6,5}}^{(3)})\Rightarrow V(G^{(3)}) = 76
\end{equation*}

%%%%%%%%%%%%%%%%%%%%%%%%%%%%

\textbf{4.2.} Заменим в таблице \ref{tab:9} нули на сумму минимального элемента строки и минимального элемента столбца. 

\begin{table}[H]
\begin{center}
	\def\tabcolsep{15pt}
	\caption{Таблица переходов}
	\label{tab:11}
	\begin{tabular}{|c||c|c|c|c|c|c|}
		\hline
		Из \textbackslash В & $1$ & $2$ & $4$ & $5$ \\
		\hhline{|=#=|=|=|=|}
		$1$ & $-$ & $-$ & $0$ & $5$ \\
		\hline
		$2$ & $-$ & $-$ & $2$ & $-$ \\
		\hline
		$4$ & $0$ & $4$ & $-$ & $-$ \\
		\hline
		$6$ & \textcolor{red}{\boldmath$8$} & $-$ & $-$ & $-$ \\
		\hline
	\end{tabular}
\end{center}
\end{table}

\begin{gather*}
\max = 8\ (6 \rightarrow 1) \\
G^{(2)} = G_{6,1}^{(2)} \cup G_{\overline{6,1}}^{(2)} \\
V(G_{\overline{6,1}}^{(2)}) = V(G^{(1)}) + 8 = 63 + 8 = 71
\end{gather*}

Вычеркнем из таблицы \ref{tab:10} строку $6$ и столбец $1$, при этом запретим переход $1 \rightarrow 5$. Найдем минимумы для каждого столбца и каждой строки.

%\begin{table}[H]
%\begin{center}
%	\def\tabcolsep{15pt}
%	\caption{Таблица переходов}
%	\label{tab:12}
%	\begin{tabular}{|c||c|c|c|c|}
%		\hline
%		Из \textbackslash В & $2$ & $4$ & $5$ & $\min$ \\
%		\hhline{|=#=|=|=|=|}
%		$1$ & $4$ & $0$ & $\infty$ & $0$ \\
%		\hline
%		$2$ & $\infty$ & $0$ & $8$ & $0$ \\
%		\hline
%		$4$ & $0$ & $\infty$ & $5$ & $0$ \\
%		\hhline{|=#=|=|=|=|}
%		$\min$ & $0$ & $0$ & $5$ & \\ 
%		\hline
%	\end{tabular}
%\end{center}
%\end{table}
%
%Вычтем из столбца $5$ минимальный элемент, равный $5$.

\begin{table}[H]
\begin{center}
	\def\tabcolsep{15pt}
	\caption{Таблица переходов}
	\label{tab:13}
	\begin{tabular}{|c||c|c|c|c|}
		\hline
		Из \textbackslash В & $2$ & $4$ & $5$ & $\min$ \\
		\hhline{|=#=|=|=|=|}
		$1$ & $4$ & $0$ & $\infty$ & $0$ \\
		\hline
		$2$ & $\infty$ & $0$ & $3$ & $0$ \\
		\hline
		$4$ & $0$ & $\infty$ & $0$ & $0$ \\
		\hhline{|=#=|=|=|=|}
		$\min$ & $0$ & $0$ & $5$ & \\ 
		\hline
	\end{tabular}
\end{center}
\end{table}

Сумма вычитаемых элементов равна $5$, следовательно:
\begin{equation*}
V(G_{6,1}^{(2)}) = V(G^{(1)}) + 5 = 68 < 71 = V(G_{\overline{6,1}}^{(2)})\Rightarrow V(G^{(2)}) = 68
\end{equation*}

\textbf{5.} Заменим в таблице \ref{tab:13} нули на сумму минимального элемента строки и минимального элемента столбца. При этом запретим переход $6 \rightarrow 1$.

\begin{table}[H]
\begin{center}
	\def\tabcolsep{15pt}
	\caption{Таблица переходов}
	\label{tab:14}
	\begin{tabular}{|c||c|c|c|}
		\hline
		Из \textbackslash В & $2$ & $4$ & $5$ \\
		\hhline{|=#=|=|=|}
		$1$ & $-$ & \textcolor{red}{\boldmath$4$} & $-$ \\
		\hline
		$2$ & $-$ & $3$ & $-$ \\
		\hline
		$4$ & \textcolor{red}{\boldmath$4$} & $-$ & $3$ \\
		\hline
	\end{tabular}
\end{center}
\end{table}

\textbf{5.1.} Первый максимум:
\begin{gather*}
\max = 4\ (1 \rightarrow 4) \\
G^{(3)} = G_{1,4}^{(3)} \cup G_{\overline{1,4}}^{(3)} \\
V(G_{\overline{1,4}}^{(3)}) = V(G^{(2)}) + 4 = 68 + 4 = 72
\end{gather*}

Вычеркнем из таблицы \ref{tab:13} строку $1$ и столбец $4$. Запретим переход $4 \rightarrow 5$. Найдем минимумы для каждого столбца и каждой строки.

\begin{table}[H]
\begin{center}
	\def\tabcolsep{15pt}
	\caption{Таблица переходов}
	\label{tab:15}
	\begin{tabular}{|c||c|c|c|}
		\hline
		Из \textbackslash В & $2$ & $5$ & $\min$ \\
		\hhline{|=#=|=|=|}
		$2$ & $\infty$ & $3$ & $3$ \\
		\hline
		$4$ & $0$ & $\infty$ & $0$ \\
		\hline
	\end{tabular}
\end{center}
\end{table}

Вычтем из строки $2$ минимальный элемент.

\begin{table}[H]
\begin{center}
	\def\tabcolsep{15pt}
	\caption{Таблица переходов}
	\label{tab:16}
	\begin{tabular}{|c||c|c|c|}
		\hline
		Из \textbackslash В & $2$ & $5$ & $\min$ \\
		\hhline{|=#=|=|=|}
		$2$ & $\infty$ & $0$ & $3$ \\
		\hline
		$4$ & $0$ & $\infty$ & $0$ \\
		\hhline{|=#=|=|=|}
		$\min$ & $0$ & $0$ & \\ 
		\hline
	\end{tabular}
\end{center}
\end{table}

Сумма вычитаемых элементов равна $3$, следовательно:
\begin{equation*}
V(G_{1,4}^{(3)}) = V(G^{(2)}) + 3 = 71 < 72 = V(G_{\overline{1,4}}^{(3)}) \Rightarrow V(G^{(3)}) = 71
\end{equation*}

\textbf{6.1.} Остаются только два не запрещенных пути: $2 \rightarrow 5$ и $4 \rightarrow 2$ общей стоимостью $0$. Общая стоимость пути равна $71$. \\

\textbf{5.2.} Второй максимум:
\begin{gather*}
\max = 4\ (4 \rightarrow 2) \\
G^{(3)} = G_{4,2}^{(3)} \cup G_{\overline{4,2}}^{(3)} \\
V(G_{\overline{4,2}}^{(3)}) = V(G^{(2)}) + 4 = 68 + 4 = 72
\end{gather*}

Вычеркнем из таблицы \ref{tab:13} строку $4$ и столбец $2$. Запретим переход $2 \rightarrow 4$. Найдем минимумы для каждого столбца и каждой строки.

\begin{table}[H]
\begin{center}
	\def\tabcolsep{15pt}
	\caption{Таблица переходов}
	\label{tab:17}
	\begin{tabular}{|c||c|c|c|}
		\hline
		Из \textbackslash В & $4$ & $5$ & $\min$ \\
		\hhline{|=#=|=|=|}
		$1$ & $0$ & $\infty$ & $0$ \\
		\hline
		$2$ & $\infty$ & $3$ & $3$ \\
		\hline
	\end{tabular}
\end{center}
\end{table}

Вычтем из строки $2$ минимальный элемент.

\begin{table}[H]
\begin{center}
	\def\tabcolsep{15pt}
	\caption{Таблица переходов}
	\label{tab:18}
	\begin{tabular}{|c||c|c|c|}
		\hline
		Из \textbackslash В & $4$ & $5$ & $\min$ \\
		\hhline{|=#=|=|=|}
		$1$ & $0$ & $\infty$ & $0$ \\
		\hline
		$2$ & $\infty$ & $0$ & $3$ \\
		\hhline{|=#=|=|=|}
		$\min$ & $0$ & $0$ & \\ 
		\hline
	\end{tabular}
\end{center}
\end{table}

Сумма вычитаемых элементов равна $3$, следовательно:
\begin{equation*}
V(G_{4,2}^{(3)}) = V(G^{(2)}) + 3 = 71 < 72 = V(G_{\overline{4,2}}^{(3)}) \Rightarrow V(G^{(3)}) = 71
\end{equation*}

\textbf{6.2.} Остаются только два не запрещенных пути: $2 \rightarrow 5$ и $1 \rightarrow 4$ общей стоимостью $0$. Общая стоимость пути равна $71$. \\

\textbf{7.} Оба полученных пути включается одинаковые переходы. Таким образом, оптимальным является путь, длина которого равна $71$:

\begin{figure}[H]
\begin{center}
	\begin{tikzpicture}[->,>=stealth',shorten >=1pt,auto,node distance=3cm,
	                    semithick]
	  \tikzstyle{every state}=[fill=white,draw=black,text=black]
	
	  \node[state] (1) {$1$};
	  \node[state] (4) [right of=1] {$4$};
	  \node[state] (2) [right of=4] {$2$};
	  \node[state] (5) [right of=2] {$5$};
	  \node[state] (3) [right of=5] {$3$};
	  \node[state] (6) [right of=3] {$6$};
	
	  \path (1) edge node {$10$} (4)
	  		(4) edge node {$7$} (2)
	  		(2) edge node {$25$} (5)
	  		(5) edge node {$7$} (3)
	  		(3) edge node {$10$} (6)
	  		(6) edge [bend right] node {$12$} (1);
	\end{tikzpicture}
	\caption{Оптимальный путь}
	\label{fig:transition_graph}
\end{center}
\end{figure}

\newpage

\begin{figure}[H]
\begin{center}
	\begin{forest}
	for tree={circle,draw, l sep=40pt, s sep=31pt}
	[50 
		[55, edge label={node[midway,left] {$5 \rightarrow 3$}}  
			[63, edge label={node[midway,left] {$3 \rightarrow 6$}} 
				[68, edge label={node[midway,left] {$6 \rightarrow 1$}}
					[71, edge label={node[midway,left] {$1 \rightarrow 4$}}
						[71, edge label={node[midway,right] {$2 \rightarrow 5$}}
							[71, edge label={node[midway,right] {$4 \rightarrow 2$}}]
						]
					]
					[72, edge label={node[midway,right] {$1 \not\rightarrow 4$}}]
				]
				[71, edge label={node[midway,right] {$6 \not\rightarrow 1$}}]
			] 
			[67, edge label={node[midway,right] {$3 \not\rightarrow 6$}}
				[67, edge label={node[midway,left] {$1 \rightarrow 6$}}
					[76, edge label={node[midway,left] {$6 \rightarrow 5$}}]
					[88, edge label={node[midway,right] {$6 \not\rightarrow 5$}}]
				]
				[73, edge label={node[midway,right] {$1 \not\rightarrow 6$}}]
			] 
		]
		[57, edge label={node[midway,right] {$5 \not\rightarrow 3$}}
			[64, edge label={node[midway,left] {$6 \rightarrow 3$}}
				[79, edge label={node[midway,left] {$3 \rightarrow 5$}}]
				[89, edge label={node[midway,right] {$3 \not\rightarrow 5$}}]
			]
			[68, edge label={node[midway,right] {$6 \not\rightarrow 3$}}
				[68, edge label={node[midway,left] {$1 \rightarrow 3$}}
					[74, edge label={node[midway,left] {$2 \rightarrow 2$}}]
					[77, edge label={node[midway,right] {$4 \not\rightarrow 4$}}]
				]
				[75, edge label={node[midway,right] {$1 \not\rightarrow 3$}}]
			]
		]
	]
	\end{forest}
	\caption{Дерево переходов}
	\label{pic:graph}
\end{center}
\end{figure}

\end{document}
